\documentclass[a4paper,14pt]{extarticle}
\usepackage[T2A]{fontenc}
\usepackage[utf8]{inputenc}

\usepackage[english,russian]{babel} %используем русский и английский языки с переносами
\usepackage{amssymb,amsfonts,amsmath,mathtext,cite,enumerate,float} %подключаем нужные пакеты расширений

\usepackage[pdftex]{graphicx, color}
\usepackage{color}
\usepackage{listings}
% подсчет числа изображений, таблиц, ссылок
\usepackage[figure, table]{totalcount}
\usepackage{totcount}
\newtotcounter{citnum} %From the package documentation
\def\oldbibitem{} \let\oldbibitem=\bibitem
\def\bibitem{\stepcounter{citnum}\oldbibitem}

\usepackage{algorithm}
\usepackage{algpseudocode}

\usepackage{pdflscape}
\usepackage{longtable}
\usepackage{multirow}
\usepackage[table,xcdraw]{xcolor}
\usepackage{float}
\usepackage{booktabs}
\usepackage{cases}
\floatname{algorithm}{Листинг}

\setlength{\parindent}{1.25cm}      % Абзацный отступ: 1.25 cm
\usepackage{indentfirst}            % 1-й абзац имеет отступ

\DeclareGraphicsExtensions{.png,.pdf,.jpg,.mps,.bmp}
\graphicspath{{pictures/chapter1/}, {pictures/chapter2/}, {pictures/chapter3/}, {pictures/chapter4/}}
\usepackage{bmpsize}
\usepackage[section]{placeins}
\usepackage[nooneline]{caption} \captionsetup[table]{justification=raggedleft} \captionsetup[figure]{justification=justified,labelsep=endash}

\usepackage[left=2cm,right=2cm,top=2cm,bottom=2cm,bindingoffset=0cm]{geometry} % Меняем поля страницы
\usepackage{textcomp,eurosym}
\usepackage{setspace}
\onehalfspacing % Полуторный интервал

\renewcommand{\lstlistingname}{Листинг}

\usepackage{changepage}

\usepackage{tikz} %для рисования графиков
\usepackage{pgfplots}
\usepackage[hidelinks]{hyperref}
\usepackage{graphicx}
\usepackage{subcaption}

\usepackage{lastpage}

% Перенос ячеек в таблице
\newcommand{\specialcell}[2][c]{%
\begin{tabular}[#1]{@{}c@{}}#2\end{tabular}}

\begin{document}
    % Улучшения
    \section{Изучение влияния параметра штрафной функции SVM классификатора}
По-умолчанию, параметр рассматривается равным $1$. В подходе \cite{modernApproach}
такой параметр предварительно настраивался был равен $0.05$.
% !! Имеет смысл рассмотреть и при меньших параметрах.

В качестве данных для изучения выберем коллекции {\it SentiRuEval-2015} ввиду их
доступности на момент проведения аналогичных соревнований 2016 года. Для
обучения моделей выбраны сбалансированные коллекции, поскольку коллекции такого
типа и большего объема позволяют достичь наилучшего резульатата. На рисунке
\ref{fig:cost} приведены измененения результатов прогонов в зависимости от
величины параметра штрафной функции классификатора SVM.

\begin{figure}[!htp] \centering
    \captionsetup[subfigure]{justification=centering}
    \begin{subfigure}[b]{0.45\textwidth}
        \includegraphics[width=\textwidth]{pics/2015_bank_balanced.png}
        \caption{BANK}
        \label{fig:bank_cost_changes}
    \end{subfigure}
    ~
    \begin{subfigure}[b]{0.45\textwidth}
        \includegraphics[width=\textwidth]{pics/2015_ttk_balanced.png}
        \caption{TCC}
        \label{fig:ttk_cost_changes}
    \end{subfigure}
    \caption{
        Влияние {\it параметра штрафной функции SVM классификатора}
        на результаты прогонов.
        Кривыми на графиках обозначается прогоны с соответсвующими номерами.
        Значение параметра измерялось в пределе $[0.1, 1]$ с шагом $0.1$.
    }
    \label{fig:cost}
\end{figure}

Основываясь на полученных результатах можно сделать вывод о проведении дальнейшего
тестирования на данных {\it SentiRuEval-2016}.
Для задачи BANK имеет смысл пересмотреть результаты при значении параметра в
диапазоне $[0.3, 0.5]$ (рис. \ref{fig:bank_cost_changes}).
Что касается задачи TCC, то здесь подходящим параметром явзяется значение в $0.3$
(рис. \ref{fig:ttk_cost_changes}).

\end{document}
