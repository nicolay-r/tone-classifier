\documentclass[a4paper,11pt]{extarticle}
\usepackage[T2A]{fontenc}
\usepackage[utf8]{inputenc}

\usepackage[english,russian]{babel} %используем русский и английский языки с переносами
\usepackage{amssymb,amsfonts,amsmath,mathtext,cite,enumerate,float} %подключаем нужные пакеты расширений

\usepackage[pdftex]{graphicx, color}
\usepackage{color}
\usepackage{listings}
% подсчет числа изображений, таблиц, ссылок
\usepackage[figure, table]{totalcount}
\usepackage{totcount}
\newtotcounter{citnum} %From the package documentation
\def\oldbibitem{} \let\oldbibitem=\bibitem
\def\bibitem{\stepcounter{citnum}\oldbibitem}

\usepackage{algorithm}
\usepackage{algpseudocode}

\usepackage{pdflscape}
\usepackage{longtable}
\usepackage{multirow}
\usepackage[table,xcdraw]{xcolor}
\usepackage{float}
\usepackage{booktabs}
\usepackage{cases}
\floatname{algorithm}{Листинг}
\linespread{1.3}

\setlength{\parindent}{1.25cm}      % Абзацный отступ: 1.25 cm
\usepackage{indentfirst}            % 1-й абзац имеет отступ

\DeclareGraphicsExtensions{.png,.pdf,.jpg,.mps,.bmp}
\graphicspath{{pictures/chapter1/}, {pictures/chapter2/}, {pictures/chapter3/}, {pictures/chapter4/}}
\usepackage{bmpsize}
\usepackage[section]{placeins}
\usepackage[nooneline]{caption} \captionsetup[table]{justification=raggedleft} \captionsetup[figure]{justification=justified,labelsep=endash}

\usepackage[left=2.5cm,right=2.5cm,top=3.6cm,bottom=3.6cm,bindingoffset=0cm]{geometry} % Меняем поля страницы
\usepackage{textcomp,eurosym}
\usepackage{setspace}
\onehalfspacing % Полуторный интервал

\renewcommand{\lstlistingname}{Листинг}

\usepackage{changepage}

\usepackage{tikz} %для рисования графиков
\usepackage{pgfplots}
\usepackage[hidelinks]{hyperref}
\usepackage{graphicx}
\usepackage{subcaption}

\usepackage{lastpage}

% Перенос ячеек в таблице
\newcommand{\specialcell}[2][c]{%
\begin{tabular}[#1]{@{}c@{}}#2\end{tabular}}

\begin{document}
    % Объявление команд
    \newcommand\twitter{{\it Twitter }}

    \begin{center}
        \bf
        КЛАССИФИЦИРУЕМ ТОНАЛЬНО ПРОСТО: ПОСТРОЕНИЕ И ПРИМЕНЕНИЕ ЛЕКСИКОНОВ ДЛЯ
        ТОНАЛЬНОЙ КЛАССИФИКАЦИИ СООБЩЕНИЙ
    \end{center}
    \begin{center}
        Русначенко Н. Л. (kolyarus@yandex.ru),
        МГТУ им. Н.Э. Баумана, Москва, Россия

        Лукашевич Н.В. (louk\_nat@mail.ru)
    \end{center}
    \begin{center}
        \bf
        SIMPLE SENTIMENT CLASSIFIER: BUILDING AND APPLYING LEXICONS FOR SENTIMENT
        CLASSIFICATION TASK
    \end{center}
    \begin{center}
        Rusnachenko N. L. (kolyarus@yandex.ru), BMSTU, Moscow, Russia

        Loukachevitch N.V. (louk\_nat@mail.ru)
    \end{center}


    % Abstract.
    \section{Abstract}
This paper describes the application of SVM classifier for sentiment
classification of Russian Twitter messages in the banking and telecommunications
domains of SentiRuEval-2016 competition. A variety of features were implemented
to improve the quality of message classification, especially sentiment score
features based on a set of sentiment lexicons. We compare the result differences
between train collection types (balanced/imbalanced) and its volumes, and
advantages of applying lexicon-based features to each type of the training
classifier modification. Before SentiRuEval-2016, the classifier was tested on
the previous year collection of the same competition (SentiRuEval-2015) to
obtain a better settings set. The created system achieved the third place at
SentiRuEval-2016 in both tasks. The experiments performed after the SentiRuEval-2016
evaluation allowed us to improve our results by searching for a better 'Cost'
parameter value of SVM classifier and extracting more information from lexicons
into new features. The final classifier achieved results close to the top results
of the competition.

{\bf Key words:} Machine Learning, SVM, Sentiment Analysis, Lexicons,
SentiRuEval-2016


    % Введение.
    \section{Введение}
% Актуальность, Проблема.
Огромное количество людей по всему миру пользуются микроблоговой сетью
{\it Twitter}.
Среди сообщений сети встречаются такие, авторы которых выражают мнениe и оценку
качества в различных сферах услуг.
Рост скорости появления информации ведет к разработке автоматических систем
тонального анализа.

% Постановка задачи.
Формат большинства сообщений сети представляет собой короткотекстовые посты.
Поэтому под задачей тональной классификации понимается оценка всего
сообщения по отношению к компаниям, в области которых это сообщение написано.
Оценка сообщения может быть положительной, нейтральной, либо негативной.
В русскоязычной сети на протяжении уже нескольких лет, наибольший интерес прикован к
{\it отзывам о банках} и {\it отзывам о телекоммуникационных компаниях}~\cite{tonalityAnalysis}.

% Описание.
В этой работе будет рассмотрен подход, основанный на использовании
словарей тональной окраски термов для устранения проблемы недостатка
данных для обучения модели.
Будут рассмотрены признаки, в том числе и на основе лексиконов, которые позволяют
существенно повысить качество классификации.

% Про результаты.
Подход протестирован на прошедших соревнованиях {\it SentiRuEval-2016}
в рамках конференции {\it Dialogue-2016},
продемонстрировав 3-e место среди всех участников~\cite{dialog2016}.
Рассмотренные в данной статье улучшения позволили приблизиться к результату
победителя.


    % Рассматриваемые подходы.
    \section{Обзор близких подходов}
    % Добавить обзор статей
    {\bf В работе \cite{severyn}} описан способ построения лексикона
    на основе метода <<удаленного контроля>>. В качестве исходных сообщений,
    авторы подхода использовали корпус сообщений сети {\it Twitter}, содержащий
    для каждого сообщения {\it метки мнений} ({\it positive} и {\it negative}).
    Такие метки легли в основу обучения контроля полярности классификатора.

    % Описание модели (постановка задачи, раскрытие термина)
    Задача контроля полярности ставится следующим образом. Пусть имеется
    размеченные данные $ \{{\bf x}_i, {\bf y}_i \}_{i=1}^{n}$, на основе
    которых необходимо построить функцию принятия решения
    $f({\bf x}) \to {\bf y}$, которая бы на основе входных параметров
    определяла бы результирующую метку сообщения.
    В частности, авторы использовали линейную модель {\it SVM} классификатора, с
    функцией предсказания следующего вида:
    \begin{equation}
        f = sign(w^T{\bf x} + b)
    \end{equation}

    Где $w$ -- весовые коэффициенты, полученные на основе обучающей коллекции;
    $b$ -- поправочный коэффициент.
    % Алгоритм построения модели
    Авторы статьи предлагают следующий подход автоматического построения
    лексикона и его использования для создания классификационной модели:
    \begin{enumerate}
        \item Составление неразмеченного корпуса сообщений $C$ сети {\it Twitter}.
        \item Для каждого сообщения $t_i \in C$ использовать подсказки
            (хэштеги, эмотиконы) для получения метки ({\it positive} и {\it negative})
            $y_i \in \{-1; +1\}$. Использование эмотиконов вида <<:-)>>, <<:-(>>
            в качестве индикатора выражения автора сообщения в целом.
        \item Извлечение биграмм и униграмм особенности сообщения $t_i$ в
            вектор ${\bf x}_i \in R^{|L|}$, где $L$ -- лексикон, состоящий из
            термов формата биграмм и униграмм;
        \item Построить классификационную модель ${\bf w}$ на основе корпуса
            $C = \{{\bf x}_i, {\bf y}_i \}_{i=1}^{N}$ следующим образом:
        \begin{equation}
            {\bf w} = \sum\limits_{i=1}^{N}\alpha_i y_i {\bf x}_i
        \end{equation}
        Здесь ${\bf x}_i$ выступают в качестве опорных векторов; $y_i$ -- их метки;
        $\alpha_i$ -- параметр краевой задачи, который вносит вклад в
        $w$ в случае когда положителен.
        \item Каждый компонент $w_j$ обученной модели $w$, соответствует компоненту $l_i$
            лексикона $L$, что устанавливает связи с ассоциативной оценкой.
    \end{enumerate}

    % Алгоритм построения лексикона
    Используемый лексикон составлен на основе \twitter корпуса {\it Emoticon140}.
    Метки для корпуса расставлялись на основе эмотиконов, содержащихся в
    тексте сообщений.
    Так, сообщения содержащие эмотиконы типа <<:)>> считались положительными,
    а <<:(>> -- отрицательными.
    Объем корпуса составляет {\it 1.6 млн. сообщений} с одинаковым распределением
    положительных и негативных сообщений.

    Для составления лексикона используется подход на основе вычисления
    {\it точечной меры взаимоинформации (PMI)}  \cite{lexiconSO}.
    Дополнительно авторами были составлены собственные лексиконы: MPQA, BingLiu, NRC.
    На этапе предварительного тестирования и настройки модели, отмечается прирост
    качества при увеличении числа используемых лексиконов.

    % Результаты
    Подход демонстрирует хорошие результаты качества работы классификационной
    модели. На соревнованиях {\it Semeval-2014} такой подход занял второе
    место.
    Применительно к коллекциям $SMS$ и $Twitter$, оценка качества работы на
    основе $F-$меры колеблется в диапазоне 66.8 -- 71\%.

    %
    % Обзор другой статьи
    %
    {\bf В работе \cite{modernApproach}} предлагаются способы построения {\it SVM}
    классификаторов для решения следующих задач классификации:
    \begin{itemize}
        \item Определения тональной оценки для всего сообщения в целом;
        \item Выявления тональности термов сообщения.
    \end{itemize}

    % Описание подхода (только для случая оценки сообщения в целом)
        % Идея с автоматической генерацией лексикона
    Ключевой идеей повышения качества классификации являются лексиконы,
    которые созданы автоматически на основе коллекции сети {\it Twitter}.
    Для этого, авторы разделили все сообщения корпуса на тональные классы с
    помощью такой метаинформации в сообщениях, как хэштеги.
    Для этого были составлены два множества хэштегов: {\it positive} и
    {\it negative}.

    Объем коллекции, на основе которой составлен лексикон, составляет {\it $775$
    тыс. сообщений}. Для распределения сообщений на классы, авторы использовали
    следующую логику:
    \begin{enumerate}
        \item Если в сообщении встречался хотя бы один хэштег из множества {\it positive}, то
            сообщение считается положительным;
        \item Если в сообщении встречается хотя бы один хэштег из {\it negative} множества, то
            сообщение считается отрицательным.
    \end{enumerate}

    Для построения лексиконов, авторы использовали метод {\it <<тональности словосочетаний>>}
    \cite{lexiconSO}.
    В качестве классификатора, авторы использовали {\it SVM} с линейным ядром, и
    параметром штрафной функции $C = 5\cdot 10^{-3}$.
    Векторизация сообщения включала в себя набор дополнительных признаков:
    % Используемые признаки
    \begin{itemize}
        \item {\bf Учет регистра:} количество слов, записанных в верхнем регистре;
        \item {\bf Учет хэштегов:} число входящих в сообщение хэштегов (слов с префиксом <<\#>>);
        \item {\bf Символьные $n$-граммы:} присутствие или отсутствие последовательности
            подряд идущих одинаковых символов длиной в 3, 4, и 5 символов;
        \item {\bf На основе лексиконов:}
            Множество всех лексиконов включает в себя три лексикона, созданных
            в ручную ({\it NRC Emotion Lexicon, MPQA, Bing Liu Lexicon)}, а также
            два автоматически сгенерированных ({\it Hashtag Sentiment Lexicon,
            Sentiment140}). В качестве термов выступали биграммы, униграммы,
            Пусть $w$ -- рассматриваемый токен, $p$ -- полярность. Тогда,
            авторами были составлены следующие признаки:% функции
            \begin{itemize}
                \item Число токенов, для которых выполнено: $score(w, p) > 0$;
                \item Суммарное значение $\sum_{w \in tweet} score(w,p)$;
                \item Вычисление максимума $\max_{w \in tweet} score(w,p)$;
                \item Учет оценки последнего токена, при условии: $score(w,p) > 0$.
            \end{itemize}
        \item {\bf Пунктуация:}
                подсчет числа последовательностей символов <<!>> и <<?>>,
                а также подсчет случаев когда они оба символа встречаются в одной
                последовательности;
                учет содержания знаков <<?>> или <<!>> в последнем терме.
        \item {\bf Эмотиконы:}
                Признак, указывающий на наличие эмотикона в конце сообщения.
        \item {\bf Удлиненные слова:} число слов, в которых символ повторяется
            более двух раз, например <<sooooo>>;
        \item {\bf Суффикс отрицания:} добавляет к слову суффикс {\it <<\_NEG>>},
            в случае, если перед ним имеется конструкция: {\it отрицание +
            знак пунктуации}.
            Под отрицанием понимаются слова вида: {\it<<no>>}, {\it <<shouldn't>>}.
            В качестве знаков пунктуации рассматриваются символы:
            <<,>>, <<.>>, <<:>>, <<;>>, <<!>>, <<?>>.
            Пример преобразования:
            \begin{center}
                \it
                \underline{shouldn't,} perfect \\
                perfect\_NEG
            \end{center}
    \end{itemize}

    % Результаты
    Среди всех команд, принимавших участие в соревновании {\it SemEval-2013
    <<Detecting Sentiment in Twitter>>}, описанный подход занял первое место как
    в задаче определения тональности отдельного терма, так и сообщения в целом.
    По задаче оценки термов, авторам подхода удалось добиться оценки
    $F_{score}$ = $69.02\%$. На задаче тонального анализа на уровне сообщения,
    лучшая оценка несколько выше: $88.93\%$.


    % Описание подхода.
    \section{Описание подхода}
    \label{sec:buildingApproachDescription}
    В области классификации сообщений методами машинного обучения, использование
    {\it SVM} классификатора (в сравнении с {\it Naive Bayes}) обусловлено результатами
    тестирования в \cite{svmAdvantages}, которые показывают преимущество SVM на униграммной
    модели обработки сообщений.\footnote{
        Использование униграммной модели упрощает процесс обработки сообщения с
        точки зрения добавления метаинформации, в том числе и на основе лексиконов.
        В текущем подходе все термы, содержащиеся во всех лексиконах, являются
        униграммами.
    }
    Для построения обучающей модели и предсказания
    тональности на ее основе, используется библиотека LibSVM \cite{svmClassifier}.

    \subsection{Обработка сообщений}
    \label{sec:buildingMsgProcessing}
    % Векторизация, ее параметры
    Процесс обработки сообщений коллекции сообщений состоит из следующих этапов:
    \begin{enumerate}
        \item Лемматизация слов сообщений с целью получения списка термов\footnote{
            Использование пакета Yandex Mystem:
            \url{http://tech.yandex.ru/mystem/}
        };

        \item Из сообщения удаляются следующие термы:
            символы <<Ретвита>> (термы со значением <<RT>>),
            имена пользователей сети {\it Twitter} (термы с префиксом <<@>>).
            Таким образом, помимо слов естественных языков в сообщении остаются
            \#хэштеги и {\it URL\hspace{1pt}}-адреса;
        \item Замена некоторых биграмм и униграмм на тональные префиксы.
            Для выполнения этого этапа, используется предварительно составленый
            список пар\footnote{
                [Ссылка на github.]
            }
            $D_{tone} = {\langle t, s\rangle}$, где $t$ -- терм, а $s$ --
            тональная оценка (<<+>> или <<-->>). На этом этапе, для каждого терма $t_i$
            сообщения $m$, такого что $t_i \in D_{tone}$ выполняется замена на соответствующую
            оценку $s$, которая становится префиксом следующего терма $t_{i+1}$.
            Пример преобразования:
            \begin{center}
                \it
                Сейчас \underline{хорошо} работать, \underline{плохо} было раньше.

                Сейчас +работать, -было раньше.
            \end{center}
        \item Для получения весовых коэффициентов термов предполагается
            использовать меру {\it tf-idf}.
    \end{enumerate}

    \subsection{Вспомогательные признаки классификации}
    \label{sec:buildingAdditionalFeatures}
    % В этот раздел вносим признаки, которые добавлялись к основной векторизации
    Помимо термов, составляющих вектор сообщения, предполагается внести
    следующие признаки:
    \begin{itemize}
        \item На основе эмотиконов: предварительно составляются два множества
        эмотиконов (положительные и отрицательные).
        Для каждого множества определяется $C$ -- суммарное число вхождений его
        элементов в рассматриваемое сообщение.
        Результирующий числовой коэффициент вычисляется по формуле: $C_+ - C_-$;

        \item Подсчет количества термов, записанных в ВЕРХНЕМ РЕГИСТРЕ;

        \item Подсчет числа знаков препинания: <<?>>, <<...>>, <<!>>;

        \item Пусть $L$ -- множество составленных лексиконов. Тогда относительно
            каждого лексикона $l_j \in L$ для сообщения $m$, вычисляется:
            \begin{gather}
                \sum\limits_{i=1}^N l_j(t_i), \text{ где } t_i \in m
            \end{gather}
            Если терм $t_i$ отсутствует в лексиконе, то в качестве коэффициента
            рассматривается $l_j(t_i) = 0$.
            Дополнительно выполняется нормализация полученного значения в
            диапазоне $\left[ -1, 1 \right]$ на основе преобразования:
            \begin{numcases}{}
                s = 1 - e^{-|x|}, x > 0  {\label{eq:norm1}}  \\
                s = - (1 - e^{-|x|}), x < 0 {\label{eq:norm2}}
            \end{numcases}
    \end{itemize}


    % Построение лексиконов.
    \subsection{Построение лексиконов}

Построение лексикона производится на основе {\it мере взаимной информации}
(англ. Pointwise Mutual Information) \cite{lexiconSO}.
\begin{gather}
    PMI(t_1, t_2) = log_2 \dfrac{P(t_1\cap t_2)}{P(t_1)\cdot P(t_2)}
\end{gather}

Лексиконы были составлены на основе следующих данных (параметры представлены
в таблице \ref{table:createdLexicons}):

\begin{enumerate}
    \item Корпуса коротких текстов на русском языке\footnote{
        \url{https://github.com/nicolay-r/tone-classifier/tree/2016_jan_contest/data/lexicons}
    } \cite{rubtsovaCollection};
    \item Сообщений сети {\it Twitter } за январь 2016 года;
    \item Тональный словарь созданный вручную экспертами \cite{expertLexicon}.
\end{enumerate}

\subsection{Построение лексиконов}

Построение лексикона производится на основе {\it мере взаимной информации}
(англ. Pointwise Mutual Information) \cite{lexiconSO}.
\begin{gather}
    PMI(t_1, t_2) = log_2 \dfrac{P(t_1\cap t_2)}{P(t_1)\cdot P(t_2)}
\end{gather}

Лексиконы были составлены на основе следующих данных (параметры представлены
в таблице \ref{table:createdLexicons}):

\begin{enumerate}
    \item Корпуса коротких текстов на русском языке\footnote{
        \url{https://github.com/nicolay-r/tone-classifier/tree/2016_jan_contest/data/lexicons}
    } \cite{rubtsovaCollection};
    \item Сообщений сети {\it Twitter } за январь 2016 года;
    \item Тональный словарь созданный вручную экспертами \cite{expertLexicon}.
\end{enumerate}

\subsection{Построение лексиконов}

Построение лексикона производится на основе {\it мере взаимной информации}
(англ. Pointwise Mutual Information) \cite{lexiconSO}.
\begin{gather}
    PMI(t_1, t_2) = log_2 \dfrac{P(t_1\cap t_2)}{P(t_1)\cdot P(t_2)}
\end{gather}

Лексиконы были составлены на основе следующих данных (параметры представлены
в таблице \ref{table:createdLexicons}):

\begin{enumerate}
    \item Корпуса коротких текстов на русском языке\footnote{
        \url{https://github.com/nicolay-r/tone-classifier/tree/2016_jan_contest/data/lexicons}
    } \cite{rubtsovaCollection};
    \item Сообщений сети {\it Twitter } за январь 2016 года;
    \item Тональный словарь созданный вручную экспертами \cite{expertLexicon}.
\end{enumerate}

\input{parts/description/tables/lexicons}




    % Тестирование на коллекции 2015 года.
    \section{Тестирование на коллекции SentiRuEval-2015}
\label{sec:test2015}
%
% Настройки разделяем на 2 класса: те, что влияют на вектор, и те что относятся
% к обучению и параметрам алгоритма.
%
Предварительное тестирование классификатора производилось на данных
соревнований 2015 года.
Настройки векторизации сообщений в предварительных прогонах представлены в
таблице \ref{table:settings}.

\begin{table}[ht!]
\centering
\caption{Настройки векторизации сообщений}
\label{table:settings}
\end{table}

%\begin{enumerate}
%    \item Использование русскоязычных термов и хэштегов;
%    \item Прогон №1 + применение тональных префиксов, использование лексиконов 1
%        и №2, а также учет всех признаков;
%    \item Прогон №2 + использование всех лексиконов (кроме №3)\footnote{
%        Применение лексикона, составленного на обучающей коллекции {\it SentiRuEval-2015}
%        года не привело к повышению качества (ввиду малого объема).
%    }.
%\end{enumerate}

В таблице \ref{table:bankResult2015}-\ref{table:tkkResult2015} приведены оценки
качества работы классификаторов в зависимости от настроек.
Процентный прирост качества вычисляется как отношение наибольшего значения оценки по
соответствующей метрике ($F_{macro}(neg, pos)$ или $F_{micro}(neg, pos)$) к
наименьшему.

\begin{table}[ht!]
\centering
\caption{Результаты тестирования (Коллекция BANK, {\it SentiRuEval-2015})}
\label{table:bankResult2015}
\begin{tabular}{|c|c|c|c|c|}
\hline
\multirow{3}{*}{№} & \multicolumn{4}{c|}{BANK --- сообщения о банковских компаниях}                                                               \\ \cline{2-5}
                   & \multicolumn{2}{c|}{Не сбалансированная коллекция} & \multicolumn{2}{c|}{Сбалансированная коллекция} \\ \cline{2-5}
                   & $F_{macro}(neg, pos)$    & $F_{micro}(neg, pos)$   & $F_{macro}(neg, pos)$  & $F_{micro}(neg, pos)$  \\ \hline
1                  & 0.3659                   & 0.4                     & {\bf 0.4206 (+15.0\%)}       & {\bf 0.458 (+14.5\%) }       \\ \hline
2                  & 0.3933                   & 0.4128                  & {\bf 0.4305 (+9.4\%) }       & {\bf 0.4718 (+14.2\%)}       \\ \hline
3                  & 0.4119                   & 0.4394                  & {\bf 0.4349 (+5.5\%) }       & {\bf 0.4792 (+9.0\%) }       \\ \hline
\end{tabular}
\end{table}


\begin{table}[ht!]
\centering
\caption{Результаты тестирования (Коллекция TKK, {\it SentiRuEval-2015})}
\label{table:tkkResult2015}
\begin{tabular}{|c|c|c|c|c|}
\hline
\multirow{3}{*}{№} & \multicolumn{4}{c|}{TKK --- сообщения о телекоммуникационных компаниях}                                    \\ \cline{2-5}
                   & \multicolumn{2}{c|}{Не сбалансированная коллекция} & \multicolumn{2}{c|}{Сбалансированная коллекция} \\ \cline{2-5}
                   & $F_{macro}(neg, pos)$    & $F_{micro}(neg, pos)$   & $F_{macro}(neg, pos)$  & $F_{micro}(neg, pos)$  \\ \hline
1                  & {\bf 0.4608 (+0.5\%)}          & {\bf 0.5172 (+2.5\%)}          & 0.4583                 & 0.5045                 \\ \hline
2                  & {\bf 0.4701 (+0.26\%)}         & {\bf 0.5207 (+2.0\%)}          & 0.4689                 & 0.5104                 \\ \hline
3                  & {\bf 0.4925 (+3.3\%) }         & {\bf 0.5378 (+3.7\%)}          & 0.4767                 & 0.5184                 \\ \hline
\end{tabular}
\end{table}



%
% Выводы
%
На основе полученных результатов было принято решение о создании {\bf расширенной
сбалансированной коллекции}: дополнение положительных и негативных классов
коллекции 2016 года соответствующими классами коллекции 2015 года, и дальнейшая
балансировка сообщениями.
Параметры расширенной сбалансированной коллекции (см. таблицу
\ref{table:extendedCollection}).

\begin{table}[ht!]
\centering
\caption{Расширенная обучающая сбалансированная коллекция (количество сообщений)}
\label{table:extendedCollection}
\begin{tabular}{|c|c|c|}
\hline
Коллекция & Объем класса & Всего \\ \hline
BANK      & 6765         & 20295 \\ \hline
TKK       & 4894         & 14682 \\ \hline
\end{tabular}
\end{table}



    % Участие в соревнованиях 2016 года.
    \newpage
\subsection{Участие в соревнованях SentiRuEval-2016}
    \subsubsection{Результаты}

    \begin{table}[!ht]
    \centering
    \caption{Результаты прогонов соревнования (задача BANK, {\it SentiRuEval-2016})}
    \label{my-label}
    \begin{tabular}{|c|c|c|c|c|}
    \hline
    \multirow{3}{*}{№} & \multicolumn{4}{c|}{BANK}                                                                                                                                                                                         \\ \cline{2-5}
                       & \multicolumn{2}{c|}{\begin{tabular}[c]{@{}c@{}}Не сбалансированная \\ коллекция (2015 год)\end{tabular}} & \multicolumn{2}{c|}{\begin{tabular}[c]{@{}c@{}}Расширенная сбалансированная \\ коллекция\end{tabular}} \\ \cline{2-5}
                       & $F_{macro}(neg, pos)$                               & $F_{micro}(neg, pos)$                              & $F_{macro}(neg, pos)$                              & $F_{micro}(neg, pos)$                             \\ \hline
    1                  & 0,384                                               & 0,4203                                             & 0,4536 (+18.1\%)                                   & 0,4982 (+18,53\%)                                 \\ \hline
    2                  & 0,3849                                              & 0,415                                              & 0,4672 (+20.9\%)                                   & 0,5029 (+21,1\%)                                  \\ \hline
    3                  & 0,3862                                              & 0,4218                                             & 0,4683 (+21.25\%)                                  & 0,5022(+19.06\%)                                  \\ \hline
    \end{tabular}
    \end{table}

    Текст

    \begin{table}[!ht]
    \centering
    \caption{Результаты прогонов соревнования (задача TKK, {\it SentiRuEval-2016})}
    \label{my-label}
    \begin{tabular}{|c|c|c|c|c|}
    \hline
    \multirow{3}{*}{№} & \multicolumn{4}{c|}{TKK}                                                                                                                                                                                          \\ \cline{2-5}
                       & \multicolumn{2}{c|}{\begin{tabular}[c]{@{}c@{}}Не сбалансированная \\ коллекция (2015 год)\end{tabular}} & \multicolumn{2}{c|}{\begin{tabular}[c]{@{}c@{}}Расширенная сбалансированная \\ коллекция\end{tabular}} \\ \cline{2-5}
                       & $F_{macro}(neg, pos)$                               & $F_{micro}(neg, pos)$                              & $F_{macro}(neg, pos)$                              & $F_{micro}(neg, pos)$                             \\ \hline
    1                  & 0,4849                                              & 0,641                                              & 0,5103 (+5.2\%)                                    & 0,6509 (+1.5\%)                                   \\ \hline
    2                  & 0,4832                                              & 0,6473                                             & 0,5231 (+8.2\%)                                    & 0,6508 (+0.5\%)                                   \\ \hline
    3                  & 0,5099                                              & 0,677 (+2.0\%)                                     & 0,5286 (+3.6\%)                                    & 0,6632                                            \\ \hline
    \end{tabular}
    \end{table}


    \begin{table}[ht!]
    \centering
    \caption{Влияние настройки параметра Cost (С=0.5) ({\it SentiRuEval-2016})}
    \label{my-label}
    \begin{tabular}{|c|c|c|c|c|}
    \hline
    \multirow{2}{*}{№} & \multicolumn{2}{c|}{\begin{tabular}[c]{@{}c@{}}BANK\\ (Расширенная сбалансированная\\ коллекция, C=0.5)\end{tabular}} & \multicolumn{2}{c|}{\begin{tabular}[c]{@{}c@{}}TTK\\ (Расширенная сбалансированная\\ коллекция, C=0.5)\end{tabular}} \\ \cline{2-5}
                       & $F_{macro}(neg, pos)$                                     & $F_{micro}(neg, pos)$                                     & $F_{macro}(neg, pos)$                                     & $F_{micro}(neg, pos)$                                    \\ \hline
    1                  & 0,4558 (+0.4\%)                                            & 0,5037 (+1.1\%)                                            & 0,5235 (+2.5\%)                                            & 0,6612 (+1.5\%)                                           \\ \hline
    2                  & 0,4795 (+2.6\%)                                            & 0,5167 (+2.7\%)                                            & 0,5338 (+2.0\%)                                            & 0,6610 (+1.5\%)                                           \\ \hline
    3                  & 0,4768 (+1.8\%)                                            & 0,5135(+2.2\%)                                             & 0,5452 (+3.1\%)                                            & 0,6733 (+1.5\%)                                           \\ \hline
    \end{tabular}
    \end{table}

    \subsubsection{Улучшение результатов}
    \subsubsection{Вывод}


    % Улучшения
    \section{Изучение влияния параметра штрафной функции SVM классификатора}
По-умолчанию, параметр рассматривается равным $1$. В подходе \cite{modernApproach}
такой параметр предварительно настраивался был равен $0.05$.
% !! Имеет смысл рассмотреть и при меньших параметрах.

В качестве данных для изучения выберем коллекции {\it SentiRuEval-2015} ввиду их
доступности на момент проведения аналогичных соревнований 2016 года. Для
обучения моделей выбраны сбалансированные коллекции, поскольку коллекции такого
типа и большего объема позволяют достичь наилучшего резульатата. На рисунке
\ref{fig:cost} приведены измененения результатов прогонов в зависимости от
величины параметра штрафной функции классификатора SVM.

\begin{figure}[!htp] \centering
    \captionsetup[subfigure]{justification=centering}
    \begin{subfigure}[b]{0.45\textwidth}
        \includegraphics[width=\textwidth]{pics/2015_bank_balanced.png}
        \caption{BANK}
        \label{fig:bank_cost_changes}
    \end{subfigure}
    ~
    \begin{subfigure}[b]{0.45\textwidth}
        \includegraphics[width=\textwidth]{pics/2015_ttk_balanced.png}
        \caption{TCC}
        \label{fig:ttk_cost_changes}
    \end{subfigure}
    \caption{
        Влияние {\it параметра штрафной функции SVM классификатора}
        на результаты прогонов.
        Кривыми на графиках обозначается прогоны с соответсвующими номерами.
        Значение параметра измерялось в пределе $[0.1, 1]$ с шагом $0.1$.
    }
    \label{fig:cost}
\end{figure}

Основываясь на полученных результатах можно сделать вывод о проведении дальнейшего
тестирования на данных {\it SentiRuEval-2016}.
Для задачи BANK имеет смысл пересмотреть результаты при значении параметра в
диапазоне $[0.3, 0.5]$ (рис. \ref{fig:bank_cost_changes}).
Что касается задачи TCC, то здесь подходящим параметром явзяется значение в $0.3$
(рис. \ref{fig:ttk_cost_changes}).


    % Список литературы
    \clearpage
    \newpage
    \bibliographystyle{styles/utf8gost705u}  %% стилевой файл для оформления по ГОСТу
    \addcontentsline{toc}{section}{\large Список Литературы}
    \bibliography{biblio}     %% имя библиографической базы (bib-файла)
\end{document}
