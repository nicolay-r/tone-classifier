\subsection{Коллекции данных}
    \label{sec:train}
    % Здесь рассказываем про коллекции, которые использовались несбалансированные для обучения коллекции
    Для обучения классификатора предполагается использовать соответствующие
    коллекции данных соревнований {\it SentiRuEval} (см. таблицу
    \ref{table:trainCollections}).
    Коллекции $I_{bank}^{16}, \hspace{0.1cm} I_{tcc}^{16}$ являются объединением
    размеченных экспертами данных за 2015 и 2016 года.

    \begin{table}[htp!]
\caption{Обучающие коллеции предоставленные организаторами.
        {\bf $N_+, N_0, N_-$} -- число сообщений положительного, нейтрального и
        негативного классов соответственно;
        {\bf $\sum$} -- общее число сообщений в коллекции;
        нейтральный класс является {\bf наиболее частотным}.
    }
\label{table:trainCollections}
\centering
\begin{tabular}{lcccc}
\hline
\multicolumn{1}{c|}{Название} & \multicolumn{1}{c|}{$N_+$} & \multicolumn{1}{c|}{$N_0$} & \multicolumn{1}{c|}{$N_-$} & $\sum$            \\ \hline
    $I_{bank}^{15}$           & 356                        & \textbf{3\hspace{2pt}482}  & 1\hspace{2pt}077           & 4\hspace{2pt}915  \\
    $I_{tcc}^{15}$            & 956                        & \textbf{2\hspace{2pt}269}  & 1\hspace{2pt}634           & 4\hspace{2pt}859  \\
    $I_{bank}^{16}$           & 1\hspace{2pt}354           & \textbf{4\hspace{2pt}870}  & 2\hspace{2pt}550           & 8\hspace{2pt}783  \\
    $I_{tcc}^{16}$            & 704                        & \textbf{6\hspace{2pt}756}  & 1\hspace{2pt}741           & 9\hspace{2pt}102  \\ \hline
\end{tabular}
\end{table}


    Поскольку в предоставляемых
    данных число тональных сообщений существенно уступает объему класса
    нейтральных сообщений, то дополнительно планируется создать {\it сбалансированную
    обучающую коллекцию}.
    В работе \cite{svmAdvantages}, применительно к классификаторам {\it
    Наивного Байеса} и {\it SVM}, отмечается существенный прирост качества при
    использовании коллекций сбалансированного типа.

    % Про балансировку коллекций в том числе.
    Для решения подобной задачи воспользуемся готовым общедоступным корпусом
    Ю.~Рубцовой\footnote{
        Корпус коротких текстов на русском языке на основе <<постов>> сети
        Twitter: \url{study.mokoron.com}
    }, в котором каждое сообщение автоматически распределено в одну из тональных
    групп:
    {\it positive} и {\it negative}.
    Объем каждого класса такой коллекции составляет {\it $\approx$ 110 тыс.
    сообщений}

    % (Как производить балансировку)
    Для построения сбалансированной коллекции требуется существенно меньшее
    число сообщений чем предлагается в тональном корпусе.
    В связи с этим, выберем небольшой процент наиболее эмоциональных сообщений:
    \begin{enumerate}
        \item Пусть имеется лексикон $l$ на основе корпуса Ю.~Рубцовой для определения
            списка наиболее эмоциональных термов.
        \item Сообщение $m$ будем считать {\it наиболее эмоциональным},
            если для него выполнено следующее условие:
            \begin{gather}
                \max\limits_{i=1..N} |l(t_i)| > B
            \end{gather}
            Где $B$ -- величина порогового значения; \hspace{0.5pt}
            $t_i$ -- термы сообщения $m$; \hspace{0.5pt}
            $N$ -- общее количество термов в сообщении $m$;
    \end{enumerate}

    Таким образом были сбалансированы коллеции таблицы \ref{table:trainCollections}.
    Параметры дополнительно составленных коллекций представлены
    в таблице \ref{table:balancedTrainCollections}.

    \begin{table}[htp!]
\centering
\caption{Сбалансированные обучающие коллекции;
    $N_*$ -- размер класса коллекции;
    $\sum$ -- общее число сообщений в коллекции.
}
\label{table:balancedTrainCollections}
\begin{tabular}{ccccc}
    \hline
    \multicolumn{1}{c|}{\multirow{2}{*}{Название}} & \multicolumn{2}{c|}{BANK}                                & \multicolumn{2}{c}{TCC}               \\ \cline{2-5}
    \multicolumn{1}{c|}{}                          & \multicolumn{1}{c|}{$N_*$} & \multicolumn{1}{c|}{$\sum$} & \multicolumn{1}{c|}{$N_*$} & $\sum$   \\ \hline
    $B_{15}$                                       & $3’400$                    & $10’400$                    & $2’269$                    & $6’888$  \\
    $B_{16}$                                       & $6’756$                    & $20’268$                    & $4’870$                    & $14’610$ \\ \hline
\end{tabular}
\end{table}


    % Тестовые коллекции
    Параметры коллекций для проведения тестирования содержатся в таблице
    \ref{table:testCollections}.
    Коллекии такого типа отличаются от обучающих тем, что все сообщения в них
    принадлежат нейтральному классу.
    Таким образом, классификационной модели требуется выделить из этих сообщений
    тональные и проставить соответсвующий класс.

    \newpage
\section{Тестирование \cite{myArticle}}

\subsection{Подготовка данных для построения классифицирующей модели}
    \subsubsection{Данные для обработки сообщений и составления признаков}

    \input{parts/russianListings}

    %
    % Список используемых эмотиконов
    %
    \begin{itemize}
        %
        % Составление эмотиконов.
        %
        \item Для составления {\it признаков на основе эмотиконов}, использовались следующие
        списки положительных и негативных термов (см. листинги
        \ref{lst:positiveEmoticons}-\ref{lst:negativeEmoticons}).

        \begin{lstlisting}[caption="Список положительных эмотиконов", label={lst:positiveEmoticons}]
":)", ":*", ":P", ":D", ":-)", ":-D", "=)", "x)", "xD", "хД"
        \end{lstlisting}
        \begin{lstlisting}[caption="Список негативных эмотиконов", label={lst:negativeEmoticons}]
":(", "D:", ":'(", ":/", ":-(", "D-:", ":-'(", "=(", "='(", "TT", "x(",
"Dx"
        \end{lstlisting}

        %
        % Список используемых стоп-слов
        %
        \item {\it Список используемых стоп слов}, в зависимости от рассматриваемой задачи,
            представлен в листингах \ref{lst:bankStopWords}-\ref{lst:tkkStopWords}.

        \begin{lstlisting}[caption="Список стоп слов для задачи {\it BANK}", label={lst:bankStopWords}]
"пол", "идти", "бы", "со", "в", "работа", "во", "вот", "грн", "три", "Ъ"
        \end{lstlisting}

        \begin{lstlisting}[caption="Список стоп слов для задачи {\it TKK}", label={lst:tkkStopWords}]
"о", "по", "из", "и", "в", "мочь"
        \end{lstlisting}

        %
        % Составление списка тональных префиксов
        %
        \item В {\it <<Приложении В>>} (листинг \ref{lst:tonePrefixes})
            рассматриваются лемматизированные слова и словосочетания, которые
            используются {\it для получения тональных префиксов}.
    \end{itemize}


    \subsection{Построение лексиконов}

Построение лексикона производится на основе {\it мере взаимной информации}
(англ. Pointwise Mutual Information) \cite{lexiconSO}.
\begin{gather}
    PMI(t_1, t_2) = log_2 \dfrac{P(t_1\cap t_2)}{P(t_1)\cdot P(t_2)}
\end{gather}

Лексиконы были составлены на основе следующих данных (параметры представлены
в таблице \ref{table:createdLexicons}):

\begin{enumerate}
    \item Корпуса коротких текстов на русском языке\footnote{
        \url{https://github.com/nicolay-r/tone-classifier/tree/2016_jan_contest/data/lexicons}
    } \cite{rubtsovaCollection};
    \item Сообщений сети {\it Twitter } за январь 2016 года;
    \item Тональный словарь созданный вручную экспертами \cite{expertLexicon}.
\end{enumerate}

\subsection{Построение лексиконов}

Построение лексикона производится на основе {\it мере взаимной информации}
(англ. Pointwise Mutual Information) \cite{lexiconSO}.
\begin{gather}
    PMI(t_1, t_2) = log_2 \dfrac{P(t_1\cap t_2)}{P(t_1)\cdot P(t_2)}
\end{gather}

Лексиконы были составлены на основе следующих данных (параметры представлены
в таблице \ref{table:createdLexicons}):

\begin{enumerate}
    \item Корпуса коротких текстов на русском языке\footnote{
        \url{https://github.com/nicolay-r/tone-classifier/tree/2016_jan_contest/data/lexicons}
    } \cite{rubtsovaCollection};
    \item Сообщений сети {\it Twitter } за январь 2016 года;
    \item Тональный словарь созданный вручную экспертами \cite{expertLexicon}.
\end{enumerate}

\input{parts/description/tables/lexicons}



    \subsubsection{Составление обучающих коллекций}

Одно из последних соревнований в этой области проводилось в 2015 году
({\it SentiRuEval-2015}) \cite{dialog2015}, данные которого находятся в открытом
доступе и содержат эталонную коллекцию.
Поэтому можно использовать коллекции {\it SentiRuEval-2015} для предварительного
тестирования.

Обучающие коллекции не являются сбалансированными, и содержат преобладающий
по объему класс нейтральных сообщений (отмечается в п. \ref{sec:tonalityCompetition}).
В связи с этим, дополнительно была произведена балансировка сообщениями,
содержащих термы $t$ с высокими по модулю значениями $SO(t)$ лексикона №1 (см. таблицу \ref{table:createdLexicons}).
Параметры коллекций для предварительного тестирования представлены в таблице
Параметры коллекций {\it SentiRuEval-2016} представлены в таблице \ref{table:train2015balanced}.

\begin{table}[!ht]
\centering
\caption{Параметры обучающих коллекций для предварительного тестирования {\it SentiRuEval-2015} (число сообщений в классах)}
\label{table:train2015}
\begin{tabular}{|c|c|c|c|c|}
\hline
\multicolumn{5}{|c|}{Несбалансированная обучающая коллекция {\it SentiRuEval-2015}}                 \\ \hline
Коллекция & \textit{Positive} & \textit{Neutral} & \textit{Negative} & Всего \\ \hline
BANK      & 356 (7,2\%)       & 3482 {\bf (70.84\%)}   & 1077 (21.29\%)    & 4915  \\ \hline
TKK       & 956 (19.67\%)     & 2269 {\bf (46.69\%)}   & 1634 (33.62\%)    & 4859  \\ \hline
\multicolumn{5}{|c|}{Сбалансированная коллекция}                             \\ \hline
Коллекция & \multicolumn{3}{c|}{Объем класса}                        & Всего \\ \hline
BANK      & \multicolumn{3}{c|}{3482}                                & 10446 \\ \hline
TKK       & \multicolumn{3}{c|}{2296}                                & 6888  \\ \hline
\end{tabular}
\end{table}

\begin{table}[!ht]
\centering
\caption{Параметры обучающих коллекций соревнования {\it SentiRuEval-2016} (число сообщений в классах)}
\label{table:train2015balanced}
\begin{tabular}{|c|c|c|c|c|}
\hline
Коллекция & \textit{Positive} & \textit{Neutral}       & \textit{Negative} & Всего \\ \hline
BANK      & 1354 (15.41\%)    & 4870 {\bf (55.4\%)}    & 2550 (29.03\%)    & 4915  \\ \hline
TKK       & 704 (7.7\%)       & 6756 {\bf (74.22\%)}   & 1741 (19.12\%)    & 4859  \\ \hline
\end{tabular}
\end{table}


    \section{Тестирование на коллекции SentiRuEval-2015}
\label{sec:test2015}
%
% Настройки разделяем на 2 класса: те, что влияют на вектор, и те что относятся
% к обучению и параметрам алгоритма.
%
Предварительное тестирование классификатора производилось на данных
соревнований 2015 года.
Настройки векторизации сообщений в предварительных прогонах представлены в
таблице \ref{table:settings}.

\begin{table}[ht!]
\centering
\caption{Настройки векторизации сообщений}
\label{table:settings}
\end{table}

%\begin{enumerate}
%    \item Использование русскоязычных термов и хэштегов;
%    \item Прогон №1 + применение тональных префиксов, использование лексиконов 1
%        и №2, а также учет всех признаков;
%    \item Прогон №2 + использование всех лексиконов (кроме №3)\footnote{
%        Применение лексикона, составленного на обучающей коллекции {\it SentiRuEval-2015}
%        года не привело к повышению качества (ввиду малого объема).
%    }.
%\end{enumerate}

В таблице \ref{table:bankResult2015}-\ref{table:tkkResult2015} приведены оценки
качества работы классификаторов в зависимости от настроек.
Процентный прирост качества вычисляется как отношение наибольшего значения оценки по
соответствующей метрике ($F_{macro}(neg, pos)$ или $F_{micro}(neg, pos)$) к
наименьшему.

\begin{table}[ht!]
\centering
\caption{Результаты тестирования (Коллекция BANK, {\it SentiRuEval-2015})}
\label{table:bankResult2015}
\begin{tabular}{|c|c|c|c|c|}
\hline
\multirow{3}{*}{№} & \multicolumn{4}{c|}{BANK --- сообщения о банковских компаниях}                                                               \\ \cline{2-5}
                   & \multicolumn{2}{c|}{Не сбалансированная коллекция} & \multicolumn{2}{c|}{Сбалансированная коллекция} \\ \cline{2-5}
                   & $F_{macro}(neg, pos)$    & $F_{micro}(neg, pos)$   & $F_{macro}(neg, pos)$  & $F_{micro}(neg, pos)$  \\ \hline
1                  & 0.3659                   & 0.4                     & {\bf 0.4206 (+15.0\%)}       & {\bf 0.458 (+14.5\%) }       \\ \hline
2                  & 0.3933                   & 0.4128                  & {\bf 0.4305 (+9.4\%) }       & {\bf 0.4718 (+14.2\%)}       \\ \hline
3                  & 0.4119                   & 0.4394                  & {\bf 0.4349 (+5.5\%) }       & {\bf 0.4792 (+9.0\%) }       \\ \hline
\end{tabular}
\end{table}


\begin{table}[ht!]
\centering
\caption{Результаты тестирования (Коллекция TKK, {\it SentiRuEval-2015})}
\label{table:tkkResult2015}
\begin{tabular}{|c|c|c|c|c|}
\hline
\multirow{3}{*}{№} & \multicolumn{4}{c|}{TKK --- сообщения о телекоммуникационных компаниях}                                    \\ \cline{2-5}
                   & \multicolumn{2}{c|}{Не сбалансированная коллекция} & \multicolumn{2}{c|}{Сбалансированная коллекция} \\ \cline{2-5}
                   & $F_{macro}(neg, pos)$    & $F_{micro}(neg, pos)$   & $F_{macro}(neg, pos)$  & $F_{micro}(neg, pos)$  \\ \hline
1                  & {\bf 0.4608 (+0.5\%)}          & {\bf 0.5172 (+2.5\%)}          & 0.4583                 & 0.5045                 \\ \hline
2                  & {\bf 0.4701 (+0.26\%)}         & {\bf 0.5207 (+2.0\%)}          & 0.4689                 & 0.5104                 \\ \hline
3                  & {\bf 0.4925 (+3.3\%) }         & {\bf 0.5378 (+3.7\%)}          & 0.4767                 & 0.5184                 \\ \hline
\end{tabular}
\end{table}



%
% Выводы
%
На основе полученных результатов было принято решение о создании {\bf расширенной
сбалансированной коллекции}: дополнение положительных и негативных классов
коллекции 2016 года соответствующими классами коллекции 2015 года, и дальнейшая
балансировка сообщениями.
Параметры расширенной сбалансированной коллекции (см. таблицу
\ref{table:extendedCollection}).

\begin{table}[ht!]
\centering
\caption{Расширенная обучающая сбалансированная коллекция (количество сообщений)}
\label{table:extendedCollection}
\begin{tabular}{|c|c|c|}
\hline
Коллекция & Объем класса & Всего \\ \hline
BANK      & 6765         & 20295 \\ \hline
TKK       & 4894         & 14682 \\ \hline
\end{tabular}
\end{table}



    \newpage
\subsection{Участие в соревнованях SentiRuEval-2016}
    \subsubsection{Результаты}

    \begin{table}[!ht]
    \centering
    \caption{Результаты прогонов соревнования (задача BANK, {\it SentiRuEval-2016})}
    \label{my-label}
    \begin{tabular}{|c|c|c|c|c|}
    \hline
    \multirow{3}{*}{№} & \multicolumn{4}{c|}{BANK}                                                                                                                                                                                         \\ \cline{2-5}
                       & \multicolumn{2}{c|}{\begin{tabular}[c]{@{}c@{}}Не сбалансированная \\ коллекция (2015 год)\end{tabular}} & \multicolumn{2}{c|}{\begin{tabular}[c]{@{}c@{}}Расширенная сбалансированная \\ коллекция\end{tabular}} \\ \cline{2-5}
                       & $F_{macro}(neg, pos)$                               & $F_{micro}(neg, pos)$                              & $F_{macro}(neg, pos)$                              & $F_{micro}(neg, pos)$                             \\ \hline
    1                  & 0,384                                               & 0,4203                                             & 0,4536 (+18.1\%)                                   & 0,4982 (+18,53\%)                                 \\ \hline
    2                  & 0,3849                                              & 0,415                                              & 0,4672 (+20.9\%)                                   & 0,5029 (+21,1\%)                                  \\ \hline
    3                  & 0,3862                                              & 0,4218                                             & 0,4683 (+21.25\%)                                  & 0,5022(+19.06\%)                                  \\ \hline
    \end{tabular}
    \end{table}

    Текст

    \begin{table}[!ht]
    \centering
    \caption{Результаты прогонов соревнования (задача TKK, {\it SentiRuEval-2016})}
    \label{my-label}
    \begin{tabular}{|c|c|c|c|c|}
    \hline
    \multirow{3}{*}{№} & \multicolumn{4}{c|}{TKK}                                                                                                                                                                                          \\ \cline{2-5}
                       & \multicolumn{2}{c|}{\begin{tabular}[c]{@{}c@{}}Не сбалансированная \\ коллекция (2015 год)\end{tabular}} & \multicolumn{2}{c|}{\begin{tabular}[c]{@{}c@{}}Расширенная сбалансированная \\ коллекция\end{tabular}} \\ \cline{2-5}
                       & $F_{macro}(neg, pos)$                               & $F_{micro}(neg, pos)$                              & $F_{macro}(neg, pos)$                              & $F_{micro}(neg, pos)$                             \\ \hline
    1                  & 0,4849                                              & 0,641                                              & 0,5103 (+5.2\%)                                    & 0,6509 (+1.5\%)                                   \\ \hline
    2                  & 0,4832                                              & 0,6473                                             & 0,5231 (+8.2\%)                                    & 0,6508 (+0.5\%)                                   \\ \hline
    3                  & 0,5099                                              & 0,677 (+2.0\%)                                     & 0,5286 (+3.6\%)                                    & 0,6632                                            \\ \hline
    \end{tabular}
    \end{table}


    \begin{table}[ht!]
    \centering
    \caption{Влияние настройки параметра Cost (С=0.5) ({\it SentiRuEval-2016})}
    \label{my-label}
    \begin{tabular}{|c|c|c|c|c|}
    \hline
    \multirow{2}{*}{№} & \multicolumn{2}{c|}{\begin{tabular}[c]{@{}c@{}}BANK\\ (Расширенная сбалансированная\\ коллекция, C=0.5)\end{tabular}} & \multicolumn{2}{c|}{\begin{tabular}[c]{@{}c@{}}TTK\\ (Расширенная сбалансированная\\ коллекция, C=0.5)\end{tabular}} \\ \cline{2-5}
                       & $F_{macro}(neg, pos)$                                     & $F_{micro}(neg, pos)$                                     & $F_{macro}(neg, pos)$                                     & $F_{micro}(neg, pos)$                                    \\ \hline
    1                  & 0,4558 (+0.4\%)                                            & 0,5037 (+1.1\%)                                            & 0,5235 (+2.5\%)                                            & 0,6612 (+1.5\%)                                           \\ \hline
    2                  & 0,4795 (+2.6\%)                                            & 0,5167 (+2.7\%)                                            & 0,5338 (+2.0\%)                                            & 0,6610 (+1.5\%)                                           \\ \hline
    3                  & 0,4768 (+1.8\%)                                            & 0,5135(+2.2\%)                                             & 0,5452 (+3.1\%)                                            & 0,6733 (+1.5\%)                                           \\ \hline
    \end{tabular}
    \end{table}

    \subsubsection{Улучшение результатов}
    \subsubsection{Вывод}



    Проверка модели осуществляется с помощью {\it эталонных коллекций}.
    Такие коллекции находятся в открытом доступе сразу после окончания
    проведения соревнований, и представляют собой размеченные данные
    тестовых коллекций.
