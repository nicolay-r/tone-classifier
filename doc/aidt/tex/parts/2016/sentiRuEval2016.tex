\section{Участие в соревнованях SentiRuEval-2016}

Посмотрим насколько изменятся результаты, если применить классификатор к
коллекции данных {\it SentiRuEval-2016}, используя настройки таблицы
\ref{table:settings}.
Для обучения классификатора применим все коллекции, представленные в
п. \ref{sec:train}.
Полученные результаты представлены в таблице \ref{table:results2016}.

\begin{table}[ht!]
\centering
\caption{Результаты тестирования на коллекции {\it SentiRuEval-2015};
   {\bf жирным шрифтом} отмечены лучший результат относительно каждой задачи
}
\label{table:results2015}
\begin{tabular}{ccccc}
\hline
\multicolumn{1}{c|}{\multirow{2}{*}{№}} & \multicolumn{2}{c|}{BANK}                                                   & \multicolumn{2}{c}{TCC}                                                  \\ \cline{2-5}
\multicolumn{1}{c|}{}                   & \multicolumn{1}{c|}{$I_{bank}^{15}$} & \multicolumn{1}{c|}{$B_{bank}^{15}$} & \multicolumn{1}{c|}{$I_{tcc}^{15}$} & \multicolumn{1}{c}{$B_{tcc}^{15}$} \\ \hline
1                                       & $36.70$                              & $41.93$                        & $46.10$                       & $45.67$                            \\
2                                       & $38.04$                              & $41.78$                        & $46.91$                       & $45.30$                            \\
3                                       & $40.32$                              & $42.80$                        & $47.82$                       & $46.97$                            \\
4                                       & $41.41$                              & $42.75$                        & $49.12$                       & $47.56$                            \\
5                                       & $41.28$                              & $43.31$                        & $49.10$                       & $47.41$                            \\
6                                       & $42.21$                              & ${\bf 45.56}$                  & ${\bf 50.12}$                 & $49.11$                            \\ \hline
\end{tabular}
\end{table}


Анализируя полученные результаты можно сказать, что увеличения числа признаков
положительно сказывается на качестве работы классификатора, независимо от
типа задачи.

Что касается использования сбалансированных обучающих коллекций, то
однозначный прирост наблюдается в задаче TCC, в частности при обучении на
$B_{tcc}^{16}$.
Для задачи BANK наилучший результат достигается при использовании $I_{bank}^{16}$.
Возможно что данные, которые использовались для расширения в $B_{bank}^{16}$
привели к переобучению, что выражается падением качества классификации.
