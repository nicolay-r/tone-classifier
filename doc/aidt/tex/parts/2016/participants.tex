\subsection{Сравнение с результатами участников}
Каждому участнику предлагалось протестировать несколько прогонов.
По каждому из участнику, из всех полученных результатов\footnote{
    Таблица с результатами прогонов всех участников соревнований, а также
    настройками прогонов:
    \url{https://docs.google.com/spreadsheets/d/1rCaklClawfnnSnyk4q8CW4zWuO3P38DSrLw_f2wyyjg/edit\#gid=0}
} были рассмотрены лучшие
и внесены в таблицу \ref{table:comparison}.
Описание подходов участников соревнований представлено в таблице
\ref{table:participants}.

Для сравнения с участниками, результаты текущего подхода в таблице
\ref{table:comparison} представлены под участником №1 в двух вариантах:
\begin{enumerate}
    \item {\bf На момент участия в соревнованиях:} настройки прогона описаны
        в таблице \ref{table:participants}, участник №1;
    \item {\bf После окончания соревнований:} наилучший результат таблицы \ref{table:cost}.
\end{enumerate}


\definecolor{Gray}{gray}{0.9}
\begin{table}[htp!]
\centering
\caption{
    Сравнение лучших прогонов участников соревнований {\it SentiRuEval-2015}
    с описанным в статье подходом;
    {\bf жирным шрифтом} отмечены наилучшие результаты для каждой задачи по
    каждой метрике;
}
\label{table:participants_results}
\begin{tabular}{ccccc}
\hline
\multicolumn{1}{l|}{\multirow{2}{*}{Участник}}                       & \multicolumn{2}{c|}{BANK}                                                         & \multicolumn{2}{c}{TCC}                                      \\ \cline{2-5}
\multicolumn{1}{l|}{}                                                & \multicolumn{1}{c|}{$F_{1-macro}^{PN}$} & \multicolumn{1}{c|}{$F_{1-micro}^{PN}$} & \multicolumn{1}{c|}{$F_{1-macro}^{PN}$} & $F_{1-micro}^{PN}$ \\ \hline
1 & 29.86 & 32.26 & 34.19 & 38.0  \\
2 & 33.54 & 36.56 & 48.29 & 53.62 \\
3 & --    & --    & 48.04 & 50.94 \\
4 & 35.98 & 34.30 & 46.70 & 50.60 \\
%5 & 21.72 & 21.41 & 12.37 & 12.26 \\
%6 & 14.69 & 17.21 & 12.95 & 19.06 \\
8 & 32.76 & 34.32 & 38.43 & 42.83 \\
9 & --    & --    & 35.27 & 37.65 \\
10&  35.2 & 33.70 & 44.77 & 52.82 \\ \hline
\cellcolor[HTML]{C0C0C0} статья & \cellcolor[HTML]{C0C0C0} \textbf{45.56} & \cellcolor[HTML]{C0C0C0} \textbf{51.36} & \cellcolor[HTML]{C0C0C0} \textbf{50.12} & \cellcolor[HTML]{C0C0C0}\textbf{54.41} \\ \hline
\end{tabular}
\end{table}

\begin{table}[htp!]
\centering
\caption{Настройки прогонов участников соревнований}
\label{table:participants}
\begin{tabular}{cp{13.2cm}}
\hline
Участник        & \multicolumn{1}{|c}{Настройки лучших прогонов}                                                                                                                                                                                                                     \\ \hline
1               & Текущая работа (статья конференции {\it Диалог-2016} \cite{myArticle})                                                                                                                                                                                                        \\
2               & Рекуррентная нейронная сетка ({\it LSTM}); в качестве признаков {\it Word2Vec}, обученный на внешней коллекции (посты и комментарии из социальных сетей) \cite{neuralNetworks}.                                                                                                                                            \\
4               & Словарные признаки + признаки мета-классификаторов (логистическая регрессия, ридж-регрессия, классификатор на основе градиентного бустинга и классификатор на основе нейронной сети) и линейный {\it SVM} в качестве классификатора.                                                                 \\
8               & Поиск эмоциональных слов по словарю (200 тыс. словоформ), правила их комбинирования на основе синтаксического анализа; применение онтологических правил, характерных для данной предметной области                                                                                                  \\
9               & {\it SVM}: униграммы, биграммы, словарь {\it SentiRuLex}, учет частей речи, многозначных слов (автоматический словарь коннотаций по новостям для TKK задачи)                                                                                                                                         \\
10              & {\it SVM}, в качестве признаков использовались униграммы, подвергшиеся преобразованиям ({\it не + слово = один признак}, множественные повторения символов заменяются двукратным; ссылки, ответы, даты, числа – заменяются паттернами и другие преобразования); подключение словаря {\it SentiRuLex}.\\ \hline
\end{tabular}
\end{table}



Практически все участники соревнований отдали предпочтения использованию
SVM классификатора.
Для улучшения качества работы наблюдаются схожие с рассматриваемой в данной
работе методики -- использование преобразований текста и
дополнительных признаков, часть из которых построена на основе лексиконов.
Так, при использовании словаря оценочной лексики {\it SentiRuLex} в подходах 9 и 10
(см. таблицу \ref{table:participants}) можно наблюдать примерную схожесть с результатами,
полученных в этой статье (участник №1).
Следует обратить внимание на высокие показания по микромере для задачи {\it TCC},
которые составляют  $F_{1-micro}^{PN} \approx 68.0$.

Исключением является участник №2, который классифицировал сообщения с помощью
нейронных сетей \cite{neuralNetworks}.
За последнее вреия произошел резкий скачок в этой области, что привело
к бурному развитию сетей во всех возможных направлениях их применения.
Авторы экспериментировали, и остановились на трех различных решениях:
\begin{enumerate}
    \item Построение сверточной нейронной сети (CNN), добавление признаков на
        основе {\it Word2Vec};
    \item Применение рекуррентной нейронной сетки (LSTM) с добавлением признаков
        на основе {\it Word2Vec}, который был обучен на внешней коллекции
        постов и комментариев социальных сетей;
    \item Объединение результатов двух предыдущих решений, и результата
        SVM с полиноминальным ядром над усредненным {\it Word2Vec}.
\end{enumerate}


Наилучший результат был достигнут при использовании рекуррентной нейронной сетки.
