\section{Тестирование на коллекции SentiRuEval-2015}
\label{sec:sentirueval2015}

Качество работы подготовленных моделей оценивается на основе $F_1$ меры:
\begin{equation}
    \label{eq:fmeasure}
    F_1 = 2 \cdot \dfrac{P \cdot R}{P + R}
\end{equation}

Такая мера позволяет одновременно учитывать результаты следующих параметров
относительно некоторого класса:
\begin{itemize}
    \item Точность ({\bf P}recision) -- количество сообщений, которое
        классификатор правильно отнес к соответствующему классу по отношению ко
        всему объему сообщений определенных системой в этот класс;
    \item Полнота ({\bf R}ecall) -- число найденных сообщений, которые
        действительно принадлежат соответствующему классу относительно всех
        сообщений соответствующего класса.
\end{itemize}

% Про макро/микро -усреднения при переходе к нескольким классам.
В случае, когда необходимо оценить качество работы по метрике $F_1$ относительно
несколькик классов, то применяются усреднения меры.
Различают {\it микро-} и {\it макро-} усреднения.
Для вычисления усредненной $F_{1}$ меры определяются
параметры полноты и точности с соответствующим усреднением относительно
интересующих нас классов, которые, в свою очередь, вычисляются на основе
таблиц контингентности.

%
% Если что, то можно вставить
%

Мaкроусреднение придает одинаковый вес каждому из усредняемых классов, в то
время как при микроусреднении вес учитывается на основе числа документов в
классе.
При использовании $F_{1}$-макро смещение среднего значения будет производиться в
сторону того класса, для которого классификатор сработал лучше; в тоже время,
при использовании $F_{1}$-микро, смещение будет произведено в сторону наибольшего
класса. \cite{micromacromeasures}

В данной задаче нас интересует качество определения тональных твитов, т.е.
сообщений соответсвующих положительному ({\it Positive}) и отрицательному
({\it Negative}) классам.
Будем рассматривать результаты с макроусреднением $F_1$ меры
(формула \ref{eq:fmacro12}).

\begin{equation}
    \label{eq:fmacro12}
    F_{1-macro}^{PN} = 2 \cdot \dfrac{P_{macro}^{PN} \cdot
        R_{macro}^{PN}}{P_{macro}^{PN} + R_{macro}^{PN}}
\end{equation}

В тестировании участвуют прогоны с настройками, представленными в таблице
\ref{table:settings}.
Все прогоны будут одновременно протестированы в двух областях:
\begin{itemize}
    \item BANK -- отзывы о банках;
    \item TCC -- отзывы о телекоммуникационных компаниях.
\end{itemize}

\begin{table}[ht!]
\centering
\caption{Настройки векторизации сообщений}
\label{table:settings}
\begin{tabular}{cccccccccccc}
\hline
\multicolumn{1}{c|}{}  & \multicolumn{1}{c|}{}      & \multicolumn{1}{c|}{}                                                       & \multicolumn{3}{c|}{$L_{1}$}                                                            & \multicolumn{3}{c|}{$L_{2}$}                                                            & \multicolumn{3}{c}{$L_{3}$}                                           \\ \cline{4-12}
\multicolumn{1}{c|}{№} & \multicolumn{1}{c|}{Terms} & \multicolumn{1}{c|}{\begin{tabular}[c]{@{}c@{}}All\\ features\end{tabular}} & \multicolumn{1}{c|}{$\sum$} & \multicolumn{1}{c|}{$max$} & \multicolumn{1}{c|}{$min$} & \multicolumn{1}{c|}{$\sum$} & \multicolumn{1}{c|}{$max$} & \multicolumn{1}{c|}{$min$} & \multicolumn{1}{c|}{$\sum$} & \multicolumn{1}{c|}{$max$} & $min$    \\ \hline
1                      & $\bullet$                  &                                                                             &                             &                             &                             &                             &                             &                             &                             &                             &           \\
2                      & $\bullet$                  & $\bullet$                                                                   &                             &                             &                             &                             &                             &                             &                             &                             &           \\
3                      & $\bullet$                  & $\bullet$                                                                   & $\bullet$                   &                             &                             &                             &                             &                             &                             &                             &           \\
4                      & $\bullet$                  & $\bullet$                                                                   & $\bullet$                   &                             &                             & $\bullet$                   &                             &                             &                             &                             &           \\
5                      & $\bullet$                  & $\bullet$                                                                   & $\bullet$                   &                             &                             & $\bullet$                   &                             &                             & $\bullet$                   &                             &           \\
6                      & $\bullet$                  & $\bullet$                                                                   & $\bullet$                   & $\bullet$                   & $\bullet$                   & $\bullet$                   & $\bullet$                   & $\bullet$                   & $\bullet$                   & $\bullet$                   & $\bullet$ \\ \hline
\end{tabular}
\end{table}


Рассмотрим качество работы классификационных моделей каждого из прогонов для
коллекций $BANK_{15}$ и $TCC_{15}$ в зависимости от типа обучающей коллекции
($I$ -- несбалансированной, $B$ -- сбалансированной).
Результаты проведенного тестирования зафиксированы в таблице \ref{table:results2015}.

\begin{table}[ht!]
\centering
\caption{Результаты тестирования на коллекции {\it SentiRuEval-2015};
   {\bf жирным шрифтом} отмечены лучший результат относительно каждой задачи
}
\label{table:results2015}
\begin{tabular}{ccccc}
\hline
\multicolumn{1}{c|}{\multirow{2}{*}{№}} & \multicolumn{2}{c|}{BANK}                                                   & \multicolumn{2}{c}{TCC}                                                  \\ \cline{2-5}
\multicolumn{1}{c|}{}                   & \multicolumn{1}{c|}{$I_{bank}^{15}$} & \multicolumn{1}{c|}{$B_{bank}^{15}$} & \multicolumn{1}{c|}{$I_{tcc}^{15}$} & \multicolumn{1}{c}{$B_{tcc}^{15}$} \\ \hline
1                                       & $36.70$                              & $41.93$                        & $46.10$                       & $45.67$                            \\
2                                       & $38.04$                              & $41.78$                        & $46.91$                       & $45.30$                            \\
3                                       & $40.32$                              & $42.80$                        & $47.82$                       & $46.97$                            \\
4                                       & $41.41$                              & $42.75$                        & $49.12$                       & $47.56$                            \\
5                                       & $41.28$                              & $43.31$                        & $49.10$                       & $47.41$                            \\
6                                       & $42.21$                              & ${\bf 45.56}$                  & ${\bf 50.12}$                 & $49.11$                            \\ \hline
\end{tabular}
\end{table}

% Вывод о преимуществе применения балансировки.

Независимо от типа тестируемой коллекции, исходя из полученных результатов,
можно сказать что добавление признаков на основе лексиконов стабильно
повышает качество классификации.
Так, начиная с прогона №3 наблюдается прирост качества (см. результаты
прогонов с №3 -- №6, таблица \ref{table:results2015}.
Таким образом, заявленный прирост качества в статье близкого подхода
\cite{modernApproach} можно наблюдать и в тональной классификации русскоязычного
Твиттера.

Наибольший результат достигается в прогоне №6, где по каждому из лексиконов таблицы
\ref{table:createdLexicons} вычисляются все признаки: сумма, минимум и максимум.
Добавление последних двух признаков в векторизацию сообщений изменило
результат качества на +2.25 для $BANK_{15}$, и на +1.02 для $TKK_{15}$, если
сравнивать с подходом №5.

Если проанализировать результат с точки зрения влияния балансировки, то
здесь она приходится кстати, если говорить о коллекции $BANK_{15}$.
Средний прирост по каждому прогону, при сравнении результатов $I_{bank}^{15}$
с $B_{bank}^{15}$ составил +3.02.

Твиты коллекции TCC классифицируются несколько лучше. Эта особенность
отмечается в \cite{tonalityAnalysis} что объясняется ухудшением ситуации на
Украине в период сбора сообщений для тестовой коллекции $BANK_{15}$.
Так, например слово <<санкции>> может нести негативный характер в тестовой
выборке, в то время как в обучающей колекции аналогичное слово является
нейтральным.

\subsection{Сравнение с результатами участников}

Если сравнить лучший полученный в статье результат с участниками соревнований\footnote{
    Результаты участников соревнований SentiRuEval-2015:
    \url{https://docs.google.com/spreadsheets/d/1IxGFhGO4zS5t356FePIMdJQ5IU6-tkpetoOzoXpGLPs/edit\#gid=0}
},
(см. таблицу \ref{table:participants_results}),
то наилучший результат среди участников по метрике $F_{1-macro}^{PN}$ на коллекции $BANK_{15}$
составляет 35.98 (участник №4), а для коллекции $TCC_{15}$ -- 48.04 (участник №3).

\definecolor{Gray}{gray}{0.9}
\begin{table}[htp!]
\centering
\caption{
    Сравнение лучших прогонов участников соревнований {\it SentiRuEval-2015}
    с описанным в статье подходом;
    {\bf жирным шрифтом} отмечены наилучшие результаты для каждой задачи по
    каждой метрике;
}
\label{table:participants_results}
\begin{tabular}{ccccc}
\hline
\multicolumn{1}{l|}{\multirow{2}{*}{Участник}}                       & \multicolumn{2}{c|}{BANK}                                                         & \multicolumn{2}{c}{TCC}                                      \\ \cline{2-5}
\multicolumn{1}{l|}{}                                                & \multicolumn{1}{c|}{$F_{1-macro}^{PN}$} & \multicolumn{1}{c|}{$F_{1-micro}^{PN}$} & \multicolumn{1}{c|}{$F_{1-macro}^{PN}$} & $F_{1-micro}^{PN}$ \\ \hline
1 & 29.86 & 32.26 & 34.19 & 38.0  \\
2 & 33.54 & 36.56 & 48.29 & 53.62 \\
3 & --    & --    & 48.04 & 50.94 \\
4 & 35.98 & 34.30 & 46.70 & 50.60 \\
%5 & 21.72 & 21.41 & 12.37 & 12.26 \\
%6 & 14.69 & 17.21 & 12.95 & 19.06 \\
8 & 32.76 & 34.32 & 38.43 & 42.83 \\
9 & --    & --    & 35.27 & 37.65 \\
10&  35.2 & 33.70 & 44.77 & 52.82 \\ \hline
\cellcolor[HTML]{C0C0C0} статья & \cellcolor[HTML]{C0C0C0} \textbf{45.56} & \cellcolor[HTML]{C0C0C0} \textbf{51.36} & \cellcolor[HTML]{C0C0C0} \textbf{50.12} & \cellcolor[HTML]{C0C0C0}\textbf{54.41} \\ \hline
\end{tabular}
\end{table}

Таким образом, текущий подход мог бы показать первое место на прошедших
соревнованиях {\it SentiRuEval-2015}.
Единственный нюанс -- необходимо было бы использовать данные для составления
лексиконов, собранные до даты проведения соревнования.
Сравнивая лучший резульатат соревнования с прогоном №6
(см. таблицу \ref{table:results2015}), отрыв относительно победителя
по метрике $F_{1-macro}^{PN}$ составляет +6.23 для задачи BANK,
и +2.08 для задачи TCC.
