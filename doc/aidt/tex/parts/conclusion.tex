\section*{Вывод}
Если сравнивать полученные оценки качества с результатами зарубежных
соревнований, то можно прийти к следующему выводу: русскоязычные сообщения
сложнее поддаются классификации.
Наилучший результат\footnote{
    С использованием схожего подхода, в котором применялись лексиконы.
    Максимальный результат составляет $55.0\%$ по обоим задачам, но с
    использованием {\it рекурентных нейронных сетей} (RNN).
}
, полученный по метрике $F_{(macro)}^{PN}$ на
прошедших соревнованиях {\it SentiRuEval-2016} составляет $53.0-54.0\%$
для рассматриваемых задач, в то время как на {\it SemEval-2013/2014} эти
показатели выше на 14.0\%.

Но несмотря на это, полученные в данной статье результаты на коллекциях {\it SentiRuEval-2015/2016}
показывают, что использование лексиконов в качестве признаков в векторизации
сообщении позволяет улучшить качество работы классификатора.
Добавление числа признаков приводит к стабильному росту качества.
Для всех задач, наибольший результат был достигнут при использовании $max$ и $min$ функций
в качестве признаков.

Что касается авторасширения обучающих коллекций, то здесь результаты меняются
в зависимости от качества предварительно подготовленных обучающих данных.
На основе полученных результатов балансировку можно рекомендовать при наличии
возможности подбора наилучшего параметра $Cost$.
Падение результатов при изменении этого параметра не так сильно
выражено на сбалансированных коллекциях.
Поэтому не исключено, что лучший результат может быть достигнут при малом
значении отступа, поскольку данных для обучения больше.
