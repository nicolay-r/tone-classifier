\section*{Заключение}
В статье был описан подход к решению задачи тональной классификации сообщений
сети {\it Twitter} с использованием лексиконов.
Они нашли свое применение в качестве дополнительных признаков в векторе
сообщений, а также для отбора наиболее тональных сообщений с целью увеличения
объема обучающих коллекций.

Качество работы классификатора было протестировано в системах анализа тональности
русского языка {\it SentiRuEval-2015} и {\it SentiRuEval-2016}.
Добавляя в сообщения признаки на их основе, а также применяя лексиконы
для расширения коллекций наиболее тональными сообщениями, удалось добиться
стабильного роста качества классификации.

На последней демонстрируется довольно высокий результат (3-е место) относительно
остальных участников тестирования.
После проведения настройки классификатора, рассматриваемый в статье подход
можно считать одним из наиболее успешных на сегодняшний день для проведения
тональной классификации сообщений различных областей русскоязычной сети
{\it Twitter}.
