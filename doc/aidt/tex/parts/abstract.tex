\section{Abstract}
This paper describes the application of SVM classifier for sentiment
classification of Russian Twitter messages in the banking and telecommunications
domains of SentiRuEval-2016 competition. A variety of features were implemented
to improve the quality of message classification, especially sentiment score
features based on a set of sentiment lexicons. We compare the result differences
between train collection types (balanced/imbalanced) and its volumes, and
advantages of applying lexicon-based features to each type of the training
classifier modification. Before SentiRuEval-2016, the classifier was tested on
the previous year collection of the same competition (SentiRuEval-2015) to
obtain a better settings set. The created system achieved the third place at
SentiRuEval-2016 in both tasks. The experiments performed after the SentiRuEval-2016
evaluation allowed us to improve our results by searching for a better 'Cost'
parameter value of SVM classifier and extracting more information from lexicons
into new features. The final classifier achieved results close to the top results
of the competition.

{\bf Key words:} Machine Learning, SVM, Sentiment Analysis, Lexicons,
SentiRuEval-2016
