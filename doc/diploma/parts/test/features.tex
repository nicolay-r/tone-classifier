\subsubsection{Данные для обработки сообщений и составления признаков}

    \input{parts/russianListings}

    %
    % Список используемых эмотиконов
    %
    \begin{itemize}
        %
        % Составление эмотиконов.
        %
        \item Для составления {\it признаков на основе эмотиконов}, использовались следующие
        списки положительных и негативных термов (см. листинги
        \ref{lst:positiveEmoticons}-\ref{lst:negativeEmoticons}).

        \begin{lstlisting}[caption="Список положительных эмотиконов", label={lst:positiveEmoticons}]
":)", ":*", ":P", ":D", ":-)", ":-D", "=)", "x)", "xD", "хД"
        \end{lstlisting}
        \begin{lstlisting}[caption="Список негативных эмотиконов", label={lst:negativeEmoticons}]
":(", "D:", ":'(", ":/", ":-(", "D-:", ":-'(", "=(", "='(", "TT", "x(",
"Dx"
        \end{lstlisting}

        %
        % Список используемых стоп-слов
        %
        \item {\it Список используемых стоп слов}, в зависимости от рассматриваемой задачи,
            представлен в листингах \ref{lst:bankStopWords}-\ref{lst:tkkStopWords}.

        \begin{lstlisting}[caption="Список стоп слов для задачи {\it BANK}", label={lst:bankStopWords}]
"пол", "идти", "бы", "со", "в", "работа", "во", "вот", "грн", "три", "Ъ"
        \end{lstlisting}

        \begin{lstlisting}[caption="Список стоп слов для задачи {\it TKK}", label={lst:tkkStopWords}]
"о", "по", "из", "и", "в", "мочь"
        \end{lstlisting}

        %
        % Составление списка тональных префиксов
        %
        \item В {\it <<Приложении В>>} (листинг \ref{lst:tonePrefixes})
            рассматриваются лемматизированные слова и словосочетания, которые
            используются {\it для получения тональных префиксов}.
    \end{itemize}
