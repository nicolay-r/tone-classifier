\input{parts/russianListings}

В листингах \ref{lst:positiveTerms}-\ref{lst:negativeTerms} представлены
наиболее эмоциональные слова {\it корпуса коротких текстров Ю. Рубцовой} \cite{rubtsovaCollection}.
Степень эмоциональности определяется на основе подхода, описанного в п.
\ref{sec:soEvaluation}.

\begin{lstlisting}[caption="Эмоционально положительные термы", label={lst:positiveTerms}]
"D", "DDD", "DDDD", "XD", "DD", "царевич", "xD", "DDDDD", "DDDDDD", "уругвай",
"#улыбнуло", "хрещатик", "баярлалаа", "#ЛУЧИРАДОСТИОТРАДОСТИ", "конгениальность",
"#рубль", "аге", "XDD", "позаимствовать", "ржач", "листаться", "бесподобный",
"Microsoft", "позитив", "реквестировать", "улыбнуть", "#музыка", "бугага",
"ехууу";
\end{lstlisting}

\begin{lstlisting}[caption="Эмоционально негативные термы", label={lst:negativeTerms}]
"теракт", "погибший", "еб**ат", "цымбаларь", "#сми", "пострадавший", "гренобль",
"цег", "#Вконтакте", "траур", "критический", "однобокость", "михаэль", "поч",
"хнык", "блинн", "таскание", "пичаливать", "грусный", "пичаливать", "печалька",
"почемууу", "покоиться", "пращать", "смертница", "навидеть", "разочарованный",
"мазерать", "о", "скорбеть";
\end{lstlisting}
