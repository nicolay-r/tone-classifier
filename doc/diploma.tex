\documentclass[a4paper,14pt]{extarticle}
\usepackage[T2A]{fontenc}
\usepackage[utf8]{inputenc}

\usepackage[english,russian]{babel} %используем русский и английский языки с переносами
\usepackage{amssymb,amsfonts,amsmath,mathtext,cite,enumerate,float} %подключаем нужные пакеты расширений

\usepackage[pdftex]{graphicx, color}
\usepackage{subfigure}
\usepackage{color}
\usepackage{listings}

\usepackage{algorithm}
\usepackage{algpseudocode}

\usepackage{pdflscape}
\usepackage{longtable}
\usepackage{multirow}
\usepackage[table,xcdraw]{xcolor}
\usepackage{float}
\usepackage{booktabs}
\usepackage{cases}
\floatname{algorithm}{Листинг}

\setlength{\parindent}{1.25cm}      % Абзацный отступ: 1.25 cm
\usepackage{indentfirst}            % 1-й абзац имеет отступ

\DeclareGraphicsExtensions{.png,.pdf,.jpg,.mps,.bmp}
\graphicspath{{pictures/chapter1/}, {pictures/chapter2/}, {pictures/chapter3/}, {pictures/chapter4/}}
\usepackage{bmpsize}
\usepackage[section]{placeins}
\usepackage[nooneline]{caption} \captionsetup[table]{justification=raggedleft} \captionsetup[figure]{justification=centering,labelsep=endash}

\usepackage[left=2cm,right=2cm,top=2cm,bottom=2cm,bindingoffset=0cm]{geometry} % Меняем поля страницы
\usepackage{textcomp,eurosym}
\usepackage{setspace}
\onehalfspacing % Полуторный интервал

\renewcommand{\lstlistingname}{Листинг}

\usepackage{changepage}

\usepackage{tikz} %для рисования графиков
\usepackage{pgfplots}
\usepackage[hidelinks]{hyperref}

% перенос ячеек в таблице
\newcommand{\specialcell}[2][c]{%
\begin{tabular}[#1]{@{}c@{}}#2\end{tabular}}

%
% Начало документа
%
\begin{document}
\renewcommand{\figurename}{Рисунок}
\begin{titlepage}
\newpage

\begin{center}
Государственное образовательное учреждение высшего профессионального образования \\
\vspace{1cm}
\Large<<Московский государственный технический университет имени Н.Э. Баумана>> \\*
(МГТУ им. Н.Э. Баумана) \\*
\hrulefill
\end{center}

\flushright{ФАКУЛЬТЕТ ИНФОРМАТИКИ И СИСТЕМ УПРАВЛЕНИЯ}
\flushright{КАФЕДРА ТЕОРЕТИЧЕСКОЙ ИНФОРМАТИКИ И КОМПЬЮТЕРНЫХ ТЕХНОЛОГИЙ}

\vspace{1em}

\begin{center}
\Large Пояснительная записка \\ к дипломному проекту на тему:
\end{center}

\vspace{2.0em}

\begin{center}
	\Large
    \textsc{ \tt{Тема дипломного проекта}}
\end{center}

\vspace{6em}

\begin{flushleft}
Студент--дипломник \hrulefill \hspace{1pt} Русначенко Н. Л. \\
\vspace{1.5em}
Научный руководитель \hrulefill \hspace{1pt} Лукашевич Н. В.\\
\vspace{1.5em}
\end{flushleft}

\vspace{\fill}

\begin{center}
Москва 2016
\end{center}

\end{titlepage}

\renewcommand{\baselinestretch}{1.5}

%
% Аннотация
%
\renewcommand{\abstractname}{{Аннотация}}
\section{Abstract}
This paper describes the application of SVM classifier for sentiment
classification of Russian Twitter messages in the banking and telecommunications
domains of SentiRuEval-2016 competition. A variety of features were implemented
to improve the quality of message classification, especially sentiment score
features based on a set of sentiment lexicons. We compare the result differences
between train collection types (balanced/imbalanced) and its volumes, and
advantages of applying lexicon-based features to each type of the training
classifier modification. Before SentiRuEval-2016, the classifier was tested on
the previous year collection of the same competition (SentiRuEval-2015) to
obtain a better settings set. The created system achieved the third place at
SentiRuEval-2016 in both tasks. The experiments performed after the SentiRuEval-2016
evaluation allowed us to improve our results by searching for a better 'Cost'
parameter value of SVM classifier and extracting more information from lexicons
into new features. The final classifier achieved results close to the top results
of the competition.

{\bf Key words:} Machine Learning, SVM, Sentiment Analysis, Lexicons,
SentiRuEval-2016

\clearpage

%
% Содержание
%
\renewcommand{\contentsname}{\centering Содержание}
\tableofcontents

%
% Введение
%
\section{Введение}
% Актуальность, Проблема.
Огромное количество людей по всему миру пользуются микроблоговой сетью
{\it Twitter}.
Среди сообщений сети встречаются такие, авторы которых выражают мнениe и оценку
качества в различных сферах услуг.
Рост скорости появления информации ведет к разработке автоматических систем
тонального анализа.

% Постановка задачи.
Формат большинства сообщений сети представляет собой короткотекстовые посты.
Поэтому под задачей тональной классификации понимается оценка всего
сообщения по отношению к компаниям, в области которых это сообщение написано.
Оценка сообщения может быть положительной, нейтральной, либо негативной.
В русскоязычной сети на протяжении уже нескольких лет, наибольший интерес прикован к
{\it отзывам о банках} и {\it отзывам о телекоммуникационных компаниях}~\cite{tonalityAnalysis}.

% Описание.
В этой работе будет рассмотрен подход, основанный на использовании
словарей тональной окраски термов для устранения проблемы недостатка
данных для обучения модели.
Будут рассмотрены признаки, в том числе и на основе лексиконов, которые позволяют
существенно повысить качество классификации.

% Про результаты.
Подход протестирован на прошедших соревнованиях {\it SentiRuEval-2016}
в рамках конференции {\it Dialogue-2016},
продемонстрировав 3-e место среди всех участников~\cite{dialog2016}.
Рассмотренные в данной статье улучшения позволили приблизиться к результату
победителя.


%
% Обзор предметной области
%
\newcommand\twitter{{\it Twitter }}

\newpage
\section{Обзор предметной области}
    \subsection{Подходы к классификации на основе методов машинного обучения}
    \subsection{Формирование вспомогательных признаков}
        \subsubsection{Использование лексиконов}
    \subsection{Метрики оценки качества модели}
    % Посмотреть обзоры из обзора соревнований.
    % Посмотреть в дппломе 2015 года.


%
% Постановка задачи
%
\newpage
\section{Постановка задачи}
    Пусть заданы коллекции сообщений $K$, для которых известно, что все сообщения
    внутри каждой из коллекций рассматриваются относительно множества компаний ---
    $C_{K_{i}}$.
    Так, рассматривая произвольную коллекцию $K_i$, требуется для каждого
    сообщения, входящего в коллекцию, определить тональную оценку по отношению к
    компаниям.
    Оценка по отношению к каждой из компаний может быть одной из
    трех типов: положительной, нейтральной, либо негативной.

    {\it SentiRuEval} --- название соревнований, которые проводятся в этой
    области. Коллекции сообщений $K$ в рамках таких соревнований называются
    {\it тестовыми}.
    Такие коллекии составлены из сообщений сети \twitter в соответствующих областях.
    На текущий момент, рассматриваются следующие области (коллекции сообщений):
    \begin{itemize}
        \item Сообщения о банковских компаниях;
        \item Сообщения о телекоммуникационных компаниях.
    \end{itemize}

    Для каждого сообщения коллекции $K_i$ известно подмножество
    множества компаний $C_{K_{i}}$, которое в нем рассматривается. Все сообщения в
    тестовых коллекциях оцениваются нейтрально (т.е. для каждой рассматриваемой
    в нем компании проставлена нейтральная оценка).
    От участника требуется построить модель тонального классификатора,
    которая бы изменила нейтральные оценки на правильные
    (подробнее см. п. \ref{sec:tonalityCompetition}).


%
% Построение классификатора
%
\newpage
\section{Описание подхода к решению задачи тонального анализа сообщений сети Twitter}
    % Описать классификатор, который планируется использовать
    Основным и единственным параметром для классификаторов на основе методов машинного
    обучения являются вектора, описывающие исходные сообщения.
    Рассмотрим процесс преобразования сообщений в вектор, а также добавим
    дополнительные признаки которые бы способствовали повышению качества
    классификации.

    \subsection{Обработка сообщений}
    % Векторизация, ее параметры
    Пусть имеется произвольная коллекция сообщений сети {\it Twitter}. Процесс
    обработки сообщений этой коллекции состоит из выполнения следующих этапов:
    \begin{enumerate}
        \item Лемматизация слов сообщений с целью получения списка термов;

        \item Все термы сообщения можно разделить на два класса: основные и
            дополнительные. Ко второму классу относятся следующие термы:
            \begin{itemize}
                \item Символы <<Ретвита>> --- термы со значением <<RT>>;
                \item Ссылки на ресурсы сети Интернет --- {\it URL\hspace{1pt}}-адреса;
                \item Имена пользователей сети {\it Twitter} --- термы с префиксом <<@>>;
                \item Хэштеги (от англ. {\it Hashtags}) --- термы с префиксом <<\#>>.
            \end{itemize}
            Все остальные термы относятся к классу основных, и на текущем этапе
            термы этого класса остаются в сообщении. Среди множества дополнительных
            термов, в сообщении остаются только хэштеги и {\it URL\hspace{1pt}}-адреса;

        \item Применение предварительно составленного {\it списка стоп слов} ---
            это термы, которые должны быть исключены из сообщения.
            % Описать подробнее (После)
        \item Замена некоторых биграмм и униграмм на тональные префиксы.
            % Описать подробнее (После)
    \end{enumerate}

    Рассмотрим последний пункт обработки сообщения подробнее.
    Для выполнения этого этапа, предполагается что предварительно составлен
    список пар $L_{tone} = {\langle t, s\rangle}$, где $t$ -- терм, а $s$ --
    тональная оценка (<<+>> или <<-->>). На этом этапе, для каждого терма сообщения $t_i \in L_{tone}$
    выполняется замена на соответствующую оценку $s$, которая становится префиксом
    следующего терма $t_{i+1}$. Пример преобразования:
    \begin{center}
        \it
        Сейчас \underline{хорошо} работать \underline{не} то что раньше.

        Сейчас +работать --то что раньше.
    \end{center}

    Среди описанных выше этапов обработки сообщений, обязательным и неизменным
    является только лемматизация слов сообщений. Использование и настройки
    остальных этапов могут быть изменены в зависимости от предпочтений
    пользователя.

    % Написать про tf-idf и про дополнительный словарь
    Для получения весовых коэффициентов термов, предполагается использовать меру {\it tf-idf}.
    Дополнительно рассмотрим {\it исскуственное} расширение исходной
    обучающей коллекции посредством информации о числе вхождений большинства
    популярных термов в корпус объемом $10^6$ сообщений. Такое расширение
    позволит оценить частоты термов на основе коллекции потенциально
    большего объема чем объем исходной обучающей коллекции.

    \subsection{Вспомогательные признаки классификации}
    % В этот раздел вносим признаки, которые добавлялись к основной векторизации
    Помимо термов, составляющих вектор сообщения, в векторизацию предполагается внести
    все признаки, перечисленные в п. \ref{sec:additionalFeatures}:
    \begin{itemize}
        \item Преобразование эмотиконов сообщения в числовой коэффициент.
        Предварительно составляются два множества эмотиконов: негативные и
        положительные. Для каждого множества определяется $C$ -- суммарное число
        вхождений его элементов в рассматриваемое сообщение.
        Результирующий числовой коэффициент вычисляется по формуле: $C_+ - C_-$;

        \item Подсчет количества термов, записанных в верхнем регистре;

        \item Подсчет числа знаков препинания: <<?>>, <<...>>, <<!>>;

        \item Относительно каждого из предварительно составленных лексиконов, вычисляется
            сумма численных коэффициентов лексикона для термов входящих в рассматриваемое
            сообщение. Если терм отсутствует в лексиконе, то в качестве коэффициента рассматривется
            нулевое значение.

            Дополнительно производится нормализация полученного значения в
            диапазоне $\left[ -1, 1 \right]$ на основе следующего преобразования:
            \begin{numcases}{}
                s = 1 - e^{|x|}, x > 0 \nonumber \\
                s = - (1 - e^{|x|}), x < 0 \nonumber
            \end{numcases}
    \end{itemize}
    \subsection{Коллекции для обучения}
    % Здесь рассказываем про коллекции, которые использовались несбалансированнные для обучения коллекции

    % Про балансировку коллекций в том числе.


%
% Программная реализация
%
\section{Руководство пользователя}
% Какие функции выполняет приложение?
% (Позволяет согласно описанному подходу прозводить семантический анализ
% сообщений, а также измерять качество работы классификатора на основе описанных
% оценочных метрик).
Приложение предоставляет пользователю следующие возможности:
\begin{enumerate}
    \item Определение тональности сообщений тестовой коллекции по отношению к
    рассматриваемым компаниями. Для каждого сообщения коллекции, классификатор
    проставляет одну оценку из множества значений $\{1, 0, -1\}$;
    \item Вычисление качества работы классификатора на сообщениях тестовой
    коллекции с помощью сревнения полученных результатов с эталонными значениями.
    В качестве оценок используются метрики качества $F_{macro}(neg, pos)$ и
    $F_{micro}(neg, pos)$;
    \item Извлекать актуальные данные из сети {\tt Twitter} с помощью
    {\tt Streaming API}, а также инструменты для построения лексиконов на их
    основе.
\end{enumerate}
% Каким целям придерживалось приложение при разработке.
Изначально приложение проектировалось с расчетом на применение в целях
исследования описанного подхода. В связи с этим, предполагается следующий
сценарий использования --- оценка работы классификатора с целью выявления новых
признаков улучшающих качество классификации.
    \subsection{Формат входных и выходных данных приложения}
    % Здесь рассматриваются:
    %  -Формат данных, которые подаются на вход для проведения семантической
    % классификации. (XML)
    %  -Формат выходных данных (оценка качества работы классификатора
    %   (опциально), XML файл с просталвленными оценками)
    Для выполнения задачи тональной классификации сообщений, классификатору
    необходимо предоставить коллекцию данных для обучения. Поскольку за последнее
    время в России в этой области активно проводятся соревнования, организаторы
    которых заранее подготавливают коллекции данных на основе сообщений сети
    {\tt Twitter}, то в качестве формата описания коллекций
    % существует формат
    используется {\tt XML} документ, структура которого приведена в листинге ...
    % листинг с документом

    %
    % Формат описания одного сообщения + нули в тестовой коллекции
    %
    Приложение предоставляет коллекции данных банковских и телекоммуникационных
    компаний.

    %
    % Выходные данные
    %
    Одним из результатов работы классификатора является {\tt XML} документ,
    содержащий сообщения тестовой коллекции с проставленными оценками.
    Пример такого документа представлен в листинге ... .

    % [Листинг] с ответами классикатора.

    В общем случае, результирующий документ
    может исключать текст соответствующего сообщения и содержать только
    идентифицирующую информацию. В данном случае текст сообщения присутствует
    ввиду удобства просмотра результата.

    Оценка работы классфикатора представлет собой текстовое сообщение,
    содержащее оценки по метрикам  $F_{macro}(neg, pos)$ и $F_{micro}(neg, pos)$,
    а также вспомогательную информацию, на основе которой эти результаты были
    вычислены. Пример вывода результата качества работы классифкатора представлен
    в листинге ... .
    % [Листинг] с оценками работы классфикатора.

    \subsection{Настройка компонентов приложения}
    % Здесь необходимо описать, какие предварительные действия требуется выполнить,
    % чтобы пользователь смог, пользоваться приложением.
    % Добавление лексиконов в хранилище
    Сборка и эксплуатация приложения проверялась под операционной системой
    {\tt Ubuntu 14.04 x64}. Изначально необходимо выполнить установку зависимостей.
    Список комманд представлен в листинге ...
    % Установка приложения [Листинг]

        \subsubsection{Инициализация обучающих коллекций}
        По-умолчанию, приложение включает в себя данные соревнований {\tt SentiRuEval}
        2015 и 2016 годов. Для инициализации коллекций в СУБД {\tt PostgreSQL},
        пользователю необходимо выполнить последовательность действий, описанных
        в листинге ...
        % Инициализация коллекций [Листинг]

        \subsubsection{Инициализация лексиконов}
        В приложении содержит по-умолчанию лексиконы, представленные в текстовом
        формате. Для применения лексиконов в классификаторе, необходимо преобразовать
        текстовые данные в хранилище {\tt PostgreSQL}. Для этого пользователю
        достаточно выполнить следующие действия:
        % Инициализация лексиконов [Листинг]

    \subsection{Работа с приложением}
    % 1) Тестирование оценки качества работы классификатора на основе
    % подготовленных обучающих/тестовых/эталонных коллекциях.
    % (Как таковой просмотр ответов не представляет собой интересов; для этих
    % целей создана test набор коллекций (по аналогии с 2015 и 2016 годом)).
        \subsubsection{Векторизация сообщений обучающей и тестовых коллекций}
        % Т.е. есть скрипты вида train_..., test_... , которые подготавливают
        % train и test коллекции. В обоих случаях пусть генерируется словарь.
        % (Нет, они не могут быть разделены). Т.е., это неразделяемые процессы

        \subsubsection{Оценка качества работы и просмотр результатов}
        %
        % Оценка качества действительно может производиться отдельно.
        % check_
        %

        \subsubsection{Создание собственных лексиконов на основе сообщений сети
            {\tt Twitter} }
        Помимо существующих в приложении лексиконов, пользователю предоставляется
        возможность создавать их c нуля. Пользователю необходимо выполнить следующую
        последовательность действий:
        \begin{enumerate}
            \item Подключиться к приему потока сообщений сети {\tt Twitter};
            \item На основе принятого потока, отфильтровать тональные сообщения
            и создать два класса сообщений: положительные и негативные;
            \item Сгенерировать лексикон на основе полученных классов.
        \end{enumerate}

        % Листинг выполнения подключения.
        % Листинг фильтрации собщений.
        % Листинг генерации лексиконов.



%
% Тестирование
%
\newpage
\section{Тестирование \cite{myArticle}}

\subsection{Подготовка данных для построения классифицирующей модели}
    \subsubsection{Данные для обработки сообщений и составления признаков}

    \input{parts/russianListings}

    %
    % Список используемых эмотиконов
    %
    \begin{itemize}
        %
        % Составление эмотиконов.
        %
        \item Для составления {\it признаков на основе эмотиконов}, использовались следующие
        списки положительных и негативных термов (см. листинги
        \ref{lst:positiveEmoticons}-\ref{lst:negativeEmoticons}).

        \begin{lstlisting}[caption="Список положительных эмотиконов", label={lst:positiveEmoticons}]
":)", ":*", ":P", ":D", ":-)", ":-D", "=)", "x)", "xD", "хД"
        \end{lstlisting}
        \begin{lstlisting}[caption="Список негативных эмотиконов", label={lst:negativeEmoticons}]
":(", "D:", ":'(", ":/", ":-(", "D-:", ":-'(", "=(", "='(", "TT", "x(",
"Dx"
        \end{lstlisting}

        %
        % Список используемых стоп-слов
        %
        \item {\it Список используемых стоп слов}, в зависимости от рассматриваемой задачи,
            представлен в листингах \ref{lst:bankStopWords}-\ref{lst:tkkStopWords}.

        \begin{lstlisting}[caption="Список стоп слов для задачи {\it BANK}", label={lst:bankStopWords}]
"пол", "идти", "бы", "со", "в", "работа", "во", "вот", "грн", "три", "Ъ"
        \end{lstlisting}

        \begin{lstlisting}[caption="Список стоп слов для задачи {\it TKK}", label={lst:tkkStopWords}]
"о", "по", "из", "и", "в", "мочь"
        \end{lstlisting}

        %
        % Составление списка тональных префиксов
        %
        \item В {\it <<Приложении В>>} (листинг \ref{lst:tonePrefixes})
            рассматриваются лемматизированные слова и словосочетания, которые
            используются {\it для получения тональных префиксов}.
    \end{itemize}


    \subsection{Построение лексиконов}

Построение лексикона производится на основе {\it мере взаимной информации}
(англ. Pointwise Mutual Information) \cite{lexiconSO}.
\begin{gather}
    PMI(t_1, t_2) = log_2 \dfrac{P(t_1\cap t_2)}{P(t_1)\cdot P(t_2)}
\end{gather}

Лексиконы были составлены на основе следующих данных (параметры представлены
в таблице \ref{table:createdLexicons}):

\begin{enumerate}
    \item Корпуса коротких текстов на русском языке\footnote{
        \url{https://github.com/nicolay-r/tone-classifier/tree/2016_jan_contest/data/lexicons}
    } \cite{rubtsovaCollection};
    \item Сообщений сети {\it Twitter } за январь 2016 года;
    \item Тональный словарь созданный вручную экспертами \cite{expertLexicon}.
\end{enumerate}

\subsection{Построение лексиконов}

Построение лексикона производится на основе {\it мере взаимной информации}
(англ. Pointwise Mutual Information) \cite{lexiconSO}.
\begin{gather}
    PMI(t_1, t_2) = log_2 \dfrac{P(t_1\cap t_2)}{P(t_1)\cdot P(t_2)}
\end{gather}

Лексиконы были составлены на основе следующих данных (параметры представлены
в таблице \ref{table:createdLexicons}):

\begin{enumerate}
    \item Корпуса коротких текстов на русском языке\footnote{
        \url{https://github.com/nicolay-r/tone-classifier/tree/2016_jan_contest/data/lexicons}
    } \cite{rubtsovaCollection};
    \item Сообщений сети {\it Twitter } за январь 2016 года;
    \item Тональный словарь созданный вручную экспертами \cite{expertLexicon}.
\end{enumerate}

\input{parts/description/tables/lexicons}



    \subsubsection{Составление обучающих коллекций}

Одно из последних соревнований в этой области проводилось в 2015 году
({\it SentiRuEval-2015}) \cite{dialog2015}, данные которого находятся в открытом
доступе и содержат эталонную коллекцию.
Поэтому можно использовать коллекции {\it SentiRuEval-2015} для предварительного
тестирования.

Обучающие коллекции не являются сбалансированными, и содержат преобладающий
по объему класс нейтральных сообщений (отмечается в п. \ref{sec:tonalityCompetition}).
В связи с этим, дополнительно была произведена балансировка сообщениями,
содержащих термы $t$ с высокими по модулю значениями $SO(t)$ лексикона №1 (см. таблицу \ref{table:createdLexicons}).
Параметры коллекций для предварительного тестирования представлены в таблице
Параметры коллекций {\it SentiRuEval-2016} представлены в таблице \ref{table:train2015balanced}.

\begin{table}[!ht]
\centering
\caption{Параметры обучающих коллекций для предварительного тестирования {\it SentiRuEval-2015} (число сообщений в классах)}
\label{table:train2015}
\begin{tabular}{|c|c|c|c|c|}
\hline
\multicolumn{5}{|c|}{Несбалансированная обучающая коллекция {\it SentiRuEval-2015}}                 \\ \hline
Коллекция & \textit{Positive} & \textit{Neutral} & \textit{Negative} & Всего \\ \hline
BANK      & 356 (7,2\%)       & 3482 {\bf (70.84\%)}   & 1077 (21.29\%)    & 4915  \\ \hline
TKK       & 956 (19.67\%)     & 2269 {\bf (46.69\%)}   & 1634 (33.62\%)    & 4859  \\ \hline
\multicolumn{5}{|c|}{Сбалансированная коллекция}                             \\ \hline
Коллекция & \multicolumn{3}{c|}{Объем класса}                        & Всего \\ \hline
BANK      & \multicolumn{3}{c|}{3482}                                & 10446 \\ \hline
TKK       & \multicolumn{3}{c|}{2296}                                & 6888  \\ \hline
\end{tabular}
\end{table}

\begin{table}[!ht]
\centering
\caption{Параметры обучающих коллекций соревнования {\it SentiRuEval-2016} (число сообщений в классах)}
\label{table:train2015balanced}
\begin{tabular}{|c|c|c|c|c|}
\hline
Коллекция & \textit{Positive} & \textit{Neutral}       & \textit{Negative} & Всего \\ \hline
BANK      & 1354 (15.41\%)    & 4870 {\bf (55.4\%)}    & 2550 (29.03\%)    & 4915  \\ \hline
TKK       & 704 (7.7\%)       & 6756 {\bf (74.22\%)}   & 1741 (19.12\%)    & 4859  \\ \hline
\end{tabular}
\end{table}


    \section{Тестирование на коллекции SentiRuEval-2015}
\label{sec:test2015}
%
% Настройки разделяем на 2 класса: те, что влияют на вектор, и те что относятся
% к обучению и параметрам алгоритма.
%
Предварительное тестирование классификатора производилось на данных
соревнований 2015 года.
Настройки векторизации сообщений в предварительных прогонах представлены в
таблице \ref{table:settings}.

\begin{table}[ht!]
\centering
\caption{Настройки векторизации сообщений}
\label{table:settings}
\end{table}

%\begin{enumerate}
%    \item Использование русскоязычных термов и хэштегов;
%    \item Прогон №1 + применение тональных префиксов, использование лексиконов 1
%        и №2, а также учет всех признаков;
%    \item Прогон №2 + использование всех лексиконов (кроме №3)\footnote{
%        Применение лексикона, составленного на обучающей коллекции {\it SentiRuEval-2015}
%        года не привело к повышению качества (ввиду малого объема).
%    }.
%\end{enumerate}

В таблице \ref{table:bankResult2015}-\ref{table:tkkResult2015} приведены оценки
качества работы классификаторов в зависимости от настроек.
Процентный прирост качества вычисляется как отношение наибольшего значения оценки по
соответствующей метрике ($F_{macro}(neg, pos)$ или $F_{micro}(neg, pos)$) к
наименьшему.

\begin{table}[ht!]
\centering
\caption{Результаты тестирования (Коллекция BANK, {\it SentiRuEval-2015})}
\label{table:bankResult2015}
\begin{tabular}{|c|c|c|c|c|}
\hline
\multirow{3}{*}{№} & \multicolumn{4}{c|}{BANK --- сообщения о банковских компаниях}                                                               \\ \cline{2-5}
                   & \multicolumn{2}{c|}{Не сбалансированная коллекция} & \multicolumn{2}{c|}{Сбалансированная коллекция} \\ \cline{2-5}
                   & $F_{macro}(neg, pos)$    & $F_{micro}(neg, pos)$   & $F_{macro}(neg, pos)$  & $F_{micro}(neg, pos)$  \\ \hline
1                  & 0.3659                   & 0.4                     & {\bf 0.4206 (+15.0\%)}       & {\bf 0.458 (+14.5\%) }       \\ \hline
2                  & 0.3933                   & 0.4128                  & {\bf 0.4305 (+9.4\%) }       & {\bf 0.4718 (+14.2\%)}       \\ \hline
3                  & 0.4119                   & 0.4394                  & {\bf 0.4349 (+5.5\%) }       & {\bf 0.4792 (+9.0\%) }       \\ \hline
\end{tabular}
\end{table}


\begin{table}[ht!]
\centering
\caption{Результаты тестирования (Коллекция TKK, {\it SentiRuEval-2015})}
\label{table:tkkResult2015}
\begin{tabular}{|c|c|c|c|c|}
\hline
\multirow{3}{*}{№} & \multicolumn{4}{c|}{TKK --- сообщения о телекоммуникационных компаниях}                                    \\ \cline{2-5}
                   & \multicolumn{2}{c|}{Не сбалансированная коллекция} & \multicolumn{2}{c|}{Сбалансированная коллекция} \\ \cline{2-5}
                   & $F_{macro}(neg, pos)$    & $F_{micro}(neg, pos)$   & $F_{macro}(neg, pos)$  & $F_{micro}(neg, pos)$  \\ \hline
1                  & {\bf 0.4608 (+0.5\%)}          & {\bf 0.5172 (+2.5\%)}          & 0.4583                 & 0.5045                 \\ \hline
2                  & {\bf 0.4701 (+0.26\%)}         & {\bf 0.5207 (+2.0\%)}          & 0.4689                 & 0.5104                 \\ \hline
3                  & {\bf 0.4925 (+3.3\%) }         & {\bf 0.5378 (+3.7\%)}          & 0.4767                 & 0.5184                 \\ \hline
\end{tabular}
\end{table}



%
% Выводы
%
На основе полученных результатов было принято решение о создании {\bf расширенной
сбалансированной коллекции}: дополнение положительных и негативных классов
коллекции 2016 года соответствующими классами коллекции 2015 года, и дальнейшая
балансировка сообщениями.
Параметры расширенной сбалансированной коллекции (см. таблицу
\ref{table:extendedCollection}).

\begin{table}[ht!]
\centering
\caption{Расширенная обучающая сбалансированная коллекция (количество сообщений)}
\label{table:extendedCollection}
\begin{tabular}{|c|c|c|}
\hline
Коллекция & Объем класса & Всего \\ \hline
BANK      & 6765         & 20295 \\ \hline
TKK       & 4894         & 14682 \\ \hline
\end{tabular}
\end{table}



    \newpage
\subsection{Участие в соревнованях SentiRuEval-2016}
    \subsubsection{Результаты}

    \begin{table}[!ht]
    \centering
    \caption{Результаты прогонов соревнования (задача BANK, {\it SentiRuEval-2016})}
    \label{my-label}
    \begin{tabular}{|c|c|c|c|c|}
    \hline
    \multirow{3}{*}{№} & \multicolumn{4}{c|}{BANK}                                                                                                                                                                                         \\ \cline{2-5}
                       & \multicolumn{2}{c|}{\begin{tabular}[c]{@{}c@{}}Не сбалансированная \\ коллекция (2015 год)\end{tabular}} & \multicolumn{2}{c|}{\begin{tabular}[c]{@{}c@{}}Расширенная сбалансированная \\ коллекция\end{tabular}} \\ \cline{2-5}
                       & $F_{macro}(neg, pos)$                               & $F_{micro}(neg, pos)$                              & $F_{macro}(neg, pos)$                              & $F_{micro}(neg, pos)$                             \\ \hline
    1                  & 0,384                                               & 0,4203                                             & 0,4536 (+18.1\%)                                   & 0,4982 (+18,53\%)                                 \\ \hline
    2                  & 0,3849                                              & 0,415                                              & 0,4672 (+20.9\%)                                   & 0,5029 (+21,1\%)                                  \\ \hline
    3                  & 0,3862                                              & 0,4218                                             & 0,4683 (+21.25\%)                                  & 0,5022(+19.06\%)                                  \\ \hline
    \end{tabular}
    \end{table}

    Текст

    \begin{table}[!ht]
    \centering
    \caption{Результаты прогонов соревнования (задача TKK, {\it SentiRuEval-2016})}
    \label{my-label}
    \begin{tabular}{|c|c|c|c|c|}
    \hline
    \multirow{3}{*}{№} & \multicolumn{4}{c|}{TKK}                                                                                                                                                                                          \\ \cline{2-5}
                       & \multicolumn{2}{c|}{\begin{tabular}[c]{@{}c@{}}Не сбалансированная \\ коллекция (2015 год)\end{tabular}} & \multicolumn{2}{c|}{\begin{tabular}[c]{@{}c@{}}Расширенная сбалансированная \\ коллекция\end{tabular}} \\ \cline{2-5}
                       & $F_{macro}(neg, pos)$                               & $F_{micro}(neg, pos)$                              & $F_{macro}(neg, pos)$                              & $F_{micro}(neg, pos)$                             \\ \hline
    1                  & 0,4849                                              & 0,641                                              & 0,5103 (+5.2\%)                                    & 0,6509 (+1.5\%)                                   \\ \hline
    2                  & 0,4832                                              & 0,6473                                             & 0,5231 (+8.2\%)                                    & 0,6508 (+0.5\%)                                   \\ \hline
    3                  & 0,5099                                              & 0,677 (+2.0\%)                                     & 0,5286 (+3.6\%)                                    & 0,6632                                            \\ \hline
    \end{tabular}
    \end{table}


    \begin{table}[ht!]
    \centering
    \caption{Влияние настройки параметра Cost (С=0.5) ({\it SentiRuEval-2016})}
    \label{my-label}
    \begin{tabular}{|c|c|c|c|c|}
    \hline
    \multirow{2}{*}{№} & \multicolumn{2}{c|}{\begin{tabular}[c]{@{}c@{}}BANK\\ (Расширенная сбалансированная\\ коллекция, C=0.5)\end{tabular}} & \multicolumn{2}{c|}{\begin{tabular}[c]{@{}c@{}}TTK\\ (Расширенная сбалансированная\\ коллекция, C=0.5)\end{tabular}} \\ \cline{2-5}
                       & $F_{macro}(neg, pos)$                                     & $F_{micro}(neg, pos)$                                     & $F_{macro}(neg, pos)$                                     & $F_{micro}(neg, pos)$                                    \\ \hline
    1                  & 0,4558 (+0.4\%)                                            & 0,5037 (+1.1\%)                                            & 0,5235 (+2.5\%)                                            & 0,6612 (+1.5\%)                                           \\ \hline
    2                  & 0,4795 (+2.6\%)                                            & 0,5167 (+2.7\%)                                            & 0,5338 (+2.0\%)                                            & 0,6610 (+1.5\%)                                           \\ \hline
    3                  & 0,4768 (+1.8\%)                                            & 0,5135(+2.2\%)                                             & 0,5452 (+3.1\%)                                            & 0,6733 (+1.5\%)                                           \\ \hline
    \end{tabular}
    \end{table}

    \subsubsection{Улучшение результатов}
    \subsubsection{Вывод}



%
% Технико-экономическое обоснование
%
\newpage
\section{Технико-экономическое обоснование}


%
% Заключение
%
\newpage
\section*{Вывод}
Если сравнивать полученные оценки качества с результатами зарубежных
соревнований, то можно прийти к следующему выводу: русскоязычные сообщения
сложнее поддаются классификации.
Наилучший результат\footnote{
    С использованием схожего подхода, в котором применялись лексиконы.
    Максимальный результат составляет $55.0\%$ по обоим задачам, но с
    использованием {\it рекурентных нейронных сетей} (RNN).
}
, полученный по метрике $F_{(macro)}^{PN}$ на
прошедших соревнованиях {\it SentiRuEval-2016} составляет $53.0-54.0\%$
для рассматриваемых задач, в то время как на {\it SemEval-2013/2014} эти
показатели выше на 14.0\%.

Но несмотря на это, полученные в данной статье результаты на коллекциях {\it SentiRuEval-2015/2016}
показывают, что использование лексиконов в качестве признаков в векторизации
сообщении позволяет улучшить качество работы классификатора.
Добавление числа признаков приводит к стабильному росту качества.
Для всех задач, наибольший результат был достигнут при использовании $max$ и $min$ функций
в качестве признаков.

Что касается авторасширения обучающих коллекций, то здесь результаты меняются
в зависимости от качества предварительно подготовленных обучающих данных.
На основе полученных результатов балансировку можно рекомендовать при наличии
возможности подбора наилучшего параметра $Cost$.
Падение результатов при изменении этого параметра не так сильно
выражено на сбалансированных коллекциях.
Поэтому не исключено, что лучший результат может быть достигнут при малом
значении отступа, поскольку данных для обучения больше.


%
% Приложения
%
\newpage
\addcontentsline{toc}{part}{Приложение А. Извлечение сообщений из социальной сети Twitter}
\section*{Приложение А. Извлечение сообщений из социальной сети Twitter}

\lstdefinestyle{python}
{
    language=python,
    keywordstyle=\bfseries,
    basicstyle=\footnotesize,
    frame=single,
    showstringspaces=false,
    morekeywords={elif, join},
}

\lstinputlisting[]{parts/appendix/twitterConsumer.py}

\newpage
\addcontentsline{toc}{part}{Приложение Б. Наиболее эмоциональные термы корпуса коротких текстов}
\section*{Приложение Б. Наиболее эмоциональные термы корпуса коротких текстов}
\input{parts/russianListings}

В листингах \ref{lst:positiveTerms}-\ref{lst:negativeTerms} представлены
наиболее эмоциональные слова {\it корпуса коротких текстров Ю. Рубцовой} \cite{rubtsovaCollection}.
Степень эмоциональности определяется на основе подхода, описанного в п.
\ref{sec:soEvaluation}.

\begin{lstlisting}[caption="Эмоционально положительные термы", label={lst:positiveTerms}]
"D", "DDD", "DDDD", "XD", "DD", "царевич", "xD", "DDDDD", "DDDDDD", "уругвай",
"#улыбнуло", "хрещатик", "баярлалаа", "#ЛУЧИРАДОСТИОТРАДОСТИ", "конгениальность",
"#рубль", "аге", "XDD", "позаимствовать", "ржач", "листаться", "бесподобный",
"Microsoft", "позитив", "реквестировать", "улыбнуть", "#музыка", "бугага",
"ехууу";
\end{lstlisting}

\begin{lstlisting}[caption="Эмоционально негативные термы", label={lst:negativeTerms}]
"теракт", "погибший", "еб**ат", "цымбаларь", "#сми", "пострадавший", "гренобль",
"цег", "#Вконтакте", "траур", "критический", "однобокость", "михаэль", "поч",
"хнык", "блинн", "таскание", "пичаливать", "грусный", "пичаливать", "печалька",
"почемууу", "покоиться", "пращать", "смертница", "навидеть", "разочарованный",
"мазерать", "о", "скорбеть";
\end{lstlisting}

\newpage
\addcontentsline{toc}{part}{Приложение B. Список используемых тональных префиксов}
\section*{Приложение B. Список используемых тональных префиксов}
\input{parts/russianListings}

\begin{lstlisting}[caption="Словосочетания используемые для составления тональных префиксов", label={lst:tonePrefixes}]
"имитировать": "-", "даже если": "-", "снижение": "-", "не назвать": "-",
"уменьшение": "-", "много": "+", "весьма": "+", "просто": "+", "сильно": "+",
"спад": "-", "все время": "+", "явный": "+", "снизить": "-", "совершенно": "+"
, "снизиться": "-", "ликвидировать": "-", "значительный": "+","ослабление": "-",
"разрушать": "-", "сильнейший": "+", "нельзя назвать": "-", "разительный": "+",
"колоссально": "+", "ничего": "-", "острый": "+", "повышать": "+",
"ослаблять": "-", "пресекать": "-", "масштабный": "+","повысить": "+",
"невообразимый": "+", "настолько": "+", "якобы": "-", "вырасти": "+",
"редкостный": "+", "сильный": "+", "чрезвычайный": "+", "имитация": "-",
"намного": "+", "заметный рост": "+", "жуть как": "+", "увеличить": "+",
"необычайно": "+", "безусловный": "+", "противодействовать": "-",
"большой": "+", "крайне": "+", "перестать": "-","увеличение": "+",
"масштабность": "+", "разрушение": "-", "безумно": "+", "абсолютный": "+",
"особенно": "+", "жутко": "+", "преодолеть": "-", "запросто": "+",
"отсутствие": "-", "рост": "+", "порядочный": "+", "усилить": "+","вполне": "+",
"не": "-", "недостаточно": "-", "избыток": "+", "лишиться": "-","отменить": "-",
"абсолютно": "+", "дикий": "+", "совсем": "+", "невероятно": "+", "очень": "+",
"лишить": "-", "утратить": "-", "нет никакой": "+", "ослабить": "-",
"запредельный": "+", "утрачивать": "-", "полный": "+", "нарастание": "+",
"по-настоящему": "+", "немыслимый": "+", "гораааздо": "+", "не просто": "+",
"ликвидация": "-", "снять": "-", "преодоление": "-", "избавиться": "-",
"повышение": "+", "падение": "-", "полностью": "+", "вырастать": "+",
"предельно": "+", "нереальный": "+", "противодействие": "-", "разрушить": "-",
"совершенный": "+", "рекордно": "+", "серьезный": "+", "снижаться": "-",
"снимать": "-", "нет": "-", "достаточно": "+", "уменьшать": "-", "отмена": "-",
"поистине": "+", "никакой": "-", "наибольший": "+", "неимоверно": "+",
"уменьшить": "-", "стопроцентный": "+", "гораздо": "+", "нереально": "+",
"отсутствовать": "-", "невиданный": "+", "запрет": "-", "великий вероятность":
"+", "увеличивать": "+", "конец": "-", "лишаться": "-", "защита от": "-",
"избавление": "-", "без": "-", "потеря": "-", "жгучий": "+", "колоссальный":
"+", "терять": "-", "утрата": "-", "самый": "+", "лишать": "-", "потерять": "-",
"совсем-совсем": "+", "дефицит": "-", "чрезвычайно": "+", "нейтрализация": "-"
, "усиливать": "+", "колоссальнейший": "+", "избавляться": "-", "усиление": "+",
\end{lstlisting}


%
% Список литературы
%
\clearpage
\newpage
\bibliographystyle{styles/utf8gost705u}  %% стилевой файл для оформления по ГОСТу
\addcontentsline{toc}{section}{\large Список Литературы}
\begin{flushleft}
\bibliography{biblio}     %% имя библиографической базы (bib-файла)
\end{flushleft}

\end{document}
