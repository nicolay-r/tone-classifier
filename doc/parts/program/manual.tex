\subsection{Руководство пользователя}
    \subsubsection{Настройка компонентов приложения}
    % Здесь необходимо описать, какие предварительные действия требуется выполнить,
    % чтобы пользователь смог, пользоваться приложением.
    % Добавление лексиконов в хранилище
    Сборка и эксплуатация приложения\footnote{Исходный код приложения:
            \url{https://github.com/nicolay-r/tone-classifier/tree/2016_jan_contest/data/lexicons}
    }
    проверялась под операционной системой
    {\it Ubuntu 14.04 x64}.
    Изначально необходимо выполнить установку зависимостей.
    Список комманд представлен в листинге \ref{lst:dependencies}.
    \lstset{style=bash}
    \lstinputlisting[caption="Установка зависимостей проекта", label={lst:dependencies}]{parts/code/init.sh}

        % Инициализация обучающих коллекций
        По-умолчанию, приложение включает в себя обучающую/тестовую/эталонную
        коллекции соревнований {\it SentiRuEval} за 2015 и 2016 года. Для
        инициализации коллекций в хранилище, пользователю необходимо выполнить
        последовательность комманд, представленных в листинге \ref{lst:collectionsInit}.
        \lstset{style=bash}
        \lstinputlisting[caption="Инициализация обучающих коллекций", label={lst:collectionsInit}]{parts/code/collectionsInit.sh}

        % Создание сбалансированных обучающих коллекций
        Применение балансировки к обучающей повышает качество оценки сообщений.
        \cite{diploma2015} В листинге \ref{lst:collectionsBalancedInit} приводится
        последовательность комманд, позволяющая создать сбалансированные коллекции
        на основе обучающих коллекций {\it SentiRuEval}, дополненных сообщениями
        из {\it корпуса коротких текстов} \cite{rubtsovaCollection}.
        \lstset{style=bash}
        \lstinputlisting[caption="Создание сбалансированных коллекций.", label={lst:collectionsBalancedInit}]{parts/code/collectionsBalancedInit.sh}

        % Инициализация лексиконов
        В приложении содержит по-умолчанию лексиконы, представленные в текстовом
        формате. Для применения лексиконов в классификаторе, необходимо преобразовать
        текстовые данные в хранилище. Для этого пользователю
        достаточно выполнить следующие действия, представленные в листинге
        \ref{lst:lexiconsInit}.
        \lstset{style=bash}
        \lstinputlisting[caption="Инициализация лексиконов по-умолчанию", label={lst:lexiconsInit}]{parts/code/lexiconsInit.sh}

    \subsubsection{Работа с приложением}

        % Векторизация сообщений обучающей и тестовых коллекций
        Для работы с приложением, пользователю необходимо перейти в каталог проекта {\tt svm}.
        Каталог содержит настройки векторизации сообщений:
        % Добавить ссылку на описание формата
        \begin{itemize}
            \item {\tt msg.conf} --- описывает настройки семантической обработки текста
                сообщений в формате {\it JSON}.
            \item {\tt features.conf} --- предоставляет настройку списока
                дополнительных признаков векторизации в формате {\it JSON}.
        \end{itemize}

        Настройки параметров данных для классификации описаны в файле {\it Makefile}.
        Каждая из настроек описывается именем, которое включает в себя
        характеристику следующих параметров (формат определения значений необходимых
        параметров представлен в листинге \ref{lst:makefile}):

        \begin{enumerate}
            \item {\tt TASK\_TYPE} -- тип решаемой задачи ({bank} или {ttk});
            \item {\tt TEST\_TABLE} -- имя тестовой коллекции в хранилище;
            \item {\tt TRAIN\_TABLE} -- имя обучающей коллекции в хранилище;
            \item {\tt MODEL\_NAME} -- имя используемой модели векторизации сообщений;
            \item {\tt ETALON\_RESULT} -- ссылка на эталонную коллекцию.
        \end{enumerate}

        \lstset{style=bash}
        \lstinputlisting[caption="Пример настройки параметров классификации в Makefile.",
            label={lst:makefile}]{parts/code/makefile.sh}

        Чтобы векторизовать содержимое обучающей и тестовых коллекций, а также
        произвести оценку качества работы классификатора,необходимо выполнить
        комманду:
        \begin{center}
            {\tt make {\{название\_настройки\}}}
        \end{center}

        По завершении процесса, вся необходимая информация по оценке качества
        работы будет отображена на экране в формате, представленном в листинге
        \ref{lst:classifierScores}.

    \subsubsection{Формат описания настроек в конфигурационных файлах}

    %
    % msg.conf
    %

    %
    % features.conf
    %


%        \subsubsection{Создание лексиконов на основе сообщений сети
%            {\tt Twitter} }
%        Помимо существующих в приложении лексиконов, пользователю предоставляется
%        возможность создавать их c нуля. Пользователю необходимо выполнить следующую
%        последовательность действий (см. листинг \ref{lst:lexiconCreation}):
%        \begin{enumerate}
%            \item Подключиться к приему потока сообщений сети {\tt Twitter}. Пользователю
%                необходимо будет предварительно предоставить ключи авторизации;
%            \item На основе принятого потока, отфильтровать тональные сообщения
%            и создать два класса сообщений: положительные и негативные;
%            \item Сгенерировать лексикон на основе полученных классов.
%        \end{enumerate}
%
%        % Листинг выполнения подключения.
%        \lstset{style=bash}
%        \lstinputlisting[caption="Построение пользовательского лексикона.", label={lst:lexiconCreationPart1}]{parts/code/lexiconCreationPart1.sh}
%
%        % Листинг фильтрации собщений.
%
    % Листинг генерации лексиконов.
