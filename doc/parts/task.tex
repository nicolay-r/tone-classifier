\newpage
\section{Постановка задачи}
%
% Что это за задача
%
Задачей данной работы является разработка методов анализа тональности сообщений
Твиттера на основе комбинирования методов машинного обучения и словарей
оценочной лексики.
Сообщение представляет собой высказывание автора относительно компаний
определенной области.
Задача анализа тональности ставится как задача классификации сообщений на три
класса: позитивный, негативный, и нейтральный.
Классификация производится по отношению к компаниям, которые упоминаются в
сообщении.

%
% Что требуется для решения
%
Для решения этой задачи необходимо:
\begin{itemize}
    \item Исследовать представление текстов сообщений в виде набора признаков для
        применения методов машинного обучения;
    \item Исследовать возможности автоматического порождения словарей оценочной
        лексики;
    \item Исследовать способы представления информации из словарей оценочной
        лексики в виде признаков для системы машинного обучения;
    \item Экспериментально проверить качество созданной системы анализа
        тональности на данных тестирований систем анализа тональности для
        русского языка {\it SentiRuEval-2015} и {\it SentiRuEval-2016}.
\end{itemize}
