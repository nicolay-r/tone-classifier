\lstdefinestyle{bash}
{
    language=bash,
    extendedchars=true,
    keywordstyle=\bfseries,
    basicstyle=\footnotesize,
    frame=single,
    texcl=true
}
\lstdefinestyle{xml}
{
    language=xml,
    extendedchars=true,
    keywordstyle=\bfseries,
    basicstyle=\footnotesize,
    frame=single,
    tabsize=2,
    texcl=true
}


\section{Руководство пользователя}
% Какие функции выполняет приложение?
% (Позволяет согласно описанному подходу прозводить семантический анализ
% сообщений, а также измерять качество работы классификатора на основе описанных
% оценочных метрик).
Приложение предоставляет пользователю следующие возможности:
\begin{enumerate}
    \item Определение тональности сообщений тестовой коллекции по отношению к
    рассматриваемым компаниями. Для каждого сообщения коллекции, классификатор
    проставляет одну оценку из множества значений $\{1, 0, -1\}$;
    \item Вычисление качества работы классификатора на сообщениях тестовой
    коллекции с помощью сревнения полученных результатов с эталонными значениями.
    В качестве оценок используются метрики качества $F_{macro}(neg, pos)$ и
    $F_{micro}(neg, pos)$;
    \item Извлекать актуальные данные из сети {\tt Twitter} с помощью
    {\tt Streaming API}, а также инструменты для построения лексиконов на их
    основе.
\end{enumerate}

% Каким целям придерживалось приложение при разработке.
Изначально приложение проектировалось с расчетом на применение в целях
исследования описанного подхода. В связи с этим, предполагается следующий
сценарий использования --- оценка работы классификатора с целью выявления новых
признаков улучшающих качество классификации.
    \subsection{Формат входных и выходных данных приложения}
    % Здесь рассматриваются:
    %  -Формат данных, которые подаются на вход для проведения семантической
    % классификации. (XML)
    %  -Формат выходных данных (оценка качества работы классификатора
    %   (опциально), XML файл с просталвленными оценками)
    Для выполнения задачи тональной классификации сообщений, классификатору
    необходимо предоставить коллекцию данных для обучения. Поскольку за последнее
    время в России в этой области активно проводятся соревнования, организаторы
    которых заранее подготавливают коллекции данных на основе сообщений сети
    {\tt Twitter}, то в качестве формата описания коллекций используется
    {\tt XML} документ, структура которого представляет собой XML документ,
    экспортированный из базы данных, структура которого рассматривается в
    [ссылка].

    %\lstset{style=xml}
    %\lstinputlisting[caption="Пример описания коллекции данных.", label={lst:collectionExample}]{parts/code/collectionExample.xml}

    %
    % Выходные данные
    %
    Одним из результатов работы классификатора является аналогичный {\tt XML}
    документ, содержащий сообщения тестовой коллекции с проставленными оценками.
    %Пример такого документа представлен в листинге \ref{lst:classifierResults}.
    %\lstset{style=xml}
    %\lstinputlisting[caption="Формат описания результатов классификатора.", label={lst:classifierResults}]{parts/code/classifierResults.xml}

    %В общем случае, результирующий документ
    %может исключать текст соответствующего сообщения и содержать только
    %идентифицирующую информацию. В данном случае текст сообщения присутствует
    %ввиду удобства просмотра результата.

    Оценка работы классфикатора представлет собой текстовое сообщение,
    содержащее оценки по метрикам  $F_{macro}(neg, pos)$ и $F_{micro}(neg, pos)$,
    а также вспомогательную информацию, на основе которой эти результаты были
    вычислены. Пример вывода результата качества работы классифкатора представлен
    в листинге \ref{lst:classifierScores}.
    \lstset{style=xml}
    \lstinputlisting[caption="Формат описания результатов классификатора.", label={lst:classifierScores}]{parts/code/classifierScores.xml}

    \subsection{Настройка компонентов приложения}
    % Здесь необходимо описать, какие предварительные действия требуется выполнить,
    % чтобы пользователь смог, пользоваться приложением.
    % Добавление лексиконов в хранилище
    Сборка и эксплуатация приложения проверялась под операционной системой
    {\tt Ubuntu 14.04 x64}. Изначально необходимо выполнить установку зависимостей.
    Список комманд представлен в листинге \ref{lst:dependencies}.
    \lstset{style=bash}
    \lstinputlisting[caption="Установка зависимостей проекта", label={lst:dependencies}]{parts/code/init.sh}

    \subsubsection{Инициализация обучающих коллекций}
        По-умолчанию, приложение включает в себя данные соревнований {\tt SentiRuEval}
        2015 и 2016 годов в формате {\tt XML}. Для инициализации коллекций в хранилище
        {\tt PostgreSQL}, пользователю необходимо выполнить последовательность
        действий, описанных в листинге \ref{lst:collectionsInit}.
        \lstset{style=bash}
        \lstinputlisting[caption="Инициализация обучающих коллекций", label={lst:collectionsInit}]{parts/code/collectionsInit.sh}

        \subsubsection{Создание сбалансированных обучающих коллекций}
        Применение балансировки к обучающей повышает качество оценки сообщений. [ссылка]
        В листинге \ref{lst:collectionsBalancedInit} приводится последовательность
        комманд, позволяющая создать сбалансированные коллекции на основе обучающих
        коллекций {\tt SentiRuEval}, дополненных сообщениями из "Корпуса коротких текстов" [ссылка]
        \lstset{style=bash}
        \lstinputlisting[caption="Создание сбалансированных коллекций.", label={lst:collectionsBalancedInit}]{parts/code/collectionsBalancedInit.sh}

        \subsubsection{Инициализация лексиконов}
        В приложении содержит по-умолчанию лексиконы, представленные в текстовом
        формате. Для применения лексиконов в классификаторе, необходимо преобразовать
        текстовые данные в хранилище {\tt PostgreSQL}. Для этого пользователю
        достаточно выполнить следующие действия, представленные в листинге \ref{lst:lexiconsInit}.
        \lstset{style=bash}
        \lstinputlisting[caption="Инициализация лексиконов по-умолчанию", label={lst:lexiconsInit}]{parts/code/lexiconsInit.sh}

    \subsection{Работа с приложением}
    % 1) Тестирование оценки качества работы классификатора на основе
    % подготовленных обучающих/тестовых/эталонных коллекциях.
    % (Как таковой просмотр ответов не представляет собой интересов; для этих
    % целей создана test набор коллекций (по аналогии с 2015 и 2016 годом)).
        \subsubsection{Векторизация сообщений обучающей и тестовых коллекций}
        \label{sec:usage_vectorize}
        % Т.е. есть скрипты вида train_..., test_... , которые подготавливают
        % train и test коллекции. В обоих случаях пусть генерируется словарь.
        % (Нет, они не могут быть разделены). Т.е., это неразделяемые процессы
        % Формат настроек в отдельном разделе, тут писать его не надо !!!
        Для работы с приложением, пользователю необходимо перейти в каталог проекта {run},
        выполнив:
        \begin{center}
            {\tt cd run}
        \end{center}

        В этом же каталогe, настройки векторизации сообщений указываются и редактируются в следующих
        конфигурационных файлах:
        \begin{itemize}
            \item {\tt msg.conf} --- описывает настройки семантической обработки текста
            сообщений. % Добавить ссылку на описание формата
            \item {\tt features.conf} --- предоставляет список дополнительных признаков векторизации. % Добавить ссылку на описание формата
        \end{itemize}

        Чтобы векторизовать содержимое обучающей и тестовых коллекций, пользователю
        необходимо выполнить комманду:
        \begin{center}
            {\tt make vectorize\_{\{\bfНастройка классификатора}\}}
        \end{center}

        Где {\{\bfНастройка классификатора}\} означает имя, которая включает в
        себя характеристику следующих параметров:
        \begin{enumerate}
            \item Имя обучающей коллекции в хранилище;
            \item Имя используемой модели векторизации сообщений ({\tt tf-idf});
            \item Имя тестовой коллекции в хранилище.
        \end{enumerate}

        \subsubsection{Оценка качества работы и просмотр результатов}
        %
        % Оценка качества действительно может производиться отдельно.
        % check_
        %
        Выполнение этой операции возможно в случае, если предварительно были
        выполнены действия п. \ref{sec:usage_vectorize}.
        Для построения результатов и оценок качества классификации, необходимо
        выполнить следующую комманду:
        \begin{center}
            {\tt make check}
        \end{center}

        После этого, вся необходимая информация будет сгенерирована во вложенном
        каталоге {\tt results}, и включать в себя следующие файлы:
        \begin{itemize}
            \item {\tt out.calc} --- результы качества работы классификатора;
            \item {\tt out.res} --- тональные оценки, проставленные
            классификатором (формат см. в листинге \ref{lst:classifierScores}).
        \end{itemize}

        Для очистки каталога с результатами необходимо использовать следующую
        комманду:
        \begin{center}
            {\tt make clean}
        \end{center}

        \subsubsection{Создание лексиконов на основе сообщений сети
            {\tt Twitter} }
        Помимо существующих в приложении лексиконов, пользователю предоставляется
        возможность создавать их c нуля. Пользователю необходимо выполнить следующую
        последовательность действий (см. листинг \ref{lst:lexiconCreation}):
        \begin{enumerate}
            \item Подключиться к приему потока сообщений сети {\tt Twitter};
            \item На основе принятого потока, отфильтровать тональные сообщения
            и создать два класса сообщений: положительные и негативные;
            \item Сгенерировать лексикон на основе полученных классов.
        \end{enumerate}

        \lstset{style=bash}
        \lstinputlisting[caption="Построение пользовательского лексикона.", label={lst:lexiconCreation}]{parts/code/lexiconCreation.sh}
        % Листинг выполнения подключения.
        % Листинг фильтрации собщений.
        % Листинг генерации лексиконов.
