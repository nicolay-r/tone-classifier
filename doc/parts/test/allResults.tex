Результаты всех участников соревнований представлены в таблице \ref{table:allResults}.
Про некоторых из участников известны настройки прогонов, которые они использовали
для получения соответствующих оценок.
Описание настроек рассматривается в таблице \ref{table:usersSettings}.\footnote{
Таблица с результатами прогонов всех участников соревнований, а также настройками прогонов:
\url{https://docs.google.com/spreadsheets/d/1rCaklClawfnnSnyk4q8CW4zWuO3P38DSrLw_f2wyyjg/edit\#gid=0}
}
Участник под номером 1 соответствует текущей работе.
Соответственно, в первой строке таблицы \ref{table:allResults} рассматривается
максимально полученный в рамках этой работы результат.
Вторая строка таблицы содержит результат, полученный во время проведения соревнований.\footnote{
Разница в результатах, записанная в формате процентов, вычисляется как отношение
результата победителя соревнований к соответвующему резульату участика №1.
}

%В таблице указаны наилучшие результаты для каждой задачи каждого из участников
%по каждой из задачи относительно показания $F_{macro}(neg, pos)$. Жирным шрифтом
%выделен участник, который достиг максимальных результатов по метрикам $F_{macro}(neg, pos)$.

\begin{table}[ht!]
\centering
\caption{
В таблице указаны наилучшие результаты для каждой задачи каждого из участников
по каждой из задачи относительно показания $F_{macro}(neg, pos)$. Жирным шрифтом
выделены результаты участника, который достиг максимальных результатов по
метрикам $F_{macro}(neg, pos)$.
}
\label{table:allResults}
\begin{tabular}{|c|c|c|c|c|}
\hline
                                                                   & \multicolumn{2}{c|}{BANK}                                                        & \multicolumn{2}{c|}{TKK}                                                                                \\ \cline{2-5}
\multirow{-2}{*}{\begin{tabular}[c]{@{}c@{}}Номер\\ участника\end{tabular}}                                                & $F_{macro}(neg, pos)$                  & $F_{micro}(neg, pos)$                   & $F_{macro}(neg, pos)$                                          & $F_{micro}(neg, pos)$                  \\ \hline
\rowcolor[HTML]{FFFFFF}
\cellcolor[HTML]{FFFFFF}{\color[HTML]{333333} }                    & {\color[HTML]{333333} 0.5239 (-5.3\%)} & {\color[HTML]{333333} 0.5514 (-6.6\%)}  & {\color[HTML]{333333} 0.5453 (-2.5\%)}                         & {\color[HTML]{333333} 0.6970 (+5.7\%)} \\ \cline{2-5}
\rowcolor[HTML]{FFFFFF}
\multirow{-2}{*}{\cellcolor[HTML]{FFFFFF}{\color[HTML]{333333} 1}} & 0.4683 (-17.8\%)                       & {\color[HTML]{333333} 0.5022 (-17.1\%)} & {\color[HTML]{333333} 0.5286 (-5.8\%)}                         & 0.6632 (+0.9\%)                        \\ \hline
\textbf{2}                                                         & \textbf{0.5517}                        & {\color[HTML]{333333} \textbf{0.5881}}  & \cellcolor[HTML]{FFFFFF}{\color[HTML]{333333} \textbf{0.5594}} & \textbf{0.6569}                        \\ \hline
3                                                                  & 0.3423                                 & 0.3524                                  & 0.3994                                                         & 0.3994                                 \\ \hline
4                                                                  & 0.3730                                 & 0.3967                                  & 0.4955                                                         & 0.6252                                 \\ \hline
5                                                                  & 0.3859                                 & 0.4640                                  & 0.3499                                                         & 0.4044                                 \\ \hline
6                                                                  & 0.2398                                 & 0.3127                                  & 0.3545                                                         & 0.5263                                 \\ \hline
7                                                                  & 0.471                                  & 0.5128                                  & 0.4842                                                         & 0.6374                                 \\ \hline
8                                                                  & 0.4492                                 & 0.4705                                  & 0.4871                                                         & 0.5745                                 \\ \hline
9                                                                  & 0.5195                                 & 0.5595                                  & 0.5489                                                         & 0.6822                                 \\ \hline
10                                                                 & 0.4659                                 & 0.5053                                  & 0.5055                                                         & 0.6254                                 \\ \hline
\end{tabular}
\end{table}

\begin{table}[ht!]
\centering
\caption{Настройки прогонов участников соревнований}
\label{table:usersSettings}
\begin{tabular}{|c|p{13cm}|}
\hline
\begin{tabular}[c]{@{}c@{}}Номер\\ участника\end{tabular} & \multicolumn{1}{c|}{Настройки прогона}                                                                                                                                                                                                                                                     \\ \hline
1               & Текущая работа. Описание подхода рассмотрено в п. \ref{sec:buildingApproachDescription}.  \\ \hline
2               & рекуррентная нейронная сетка ({\it LSTM}). В качестве признаков {\it WORD2VEC}, обученный на внешней коллекции. (Посты и комментарии из социальных сетей)                                                                                                                              \\ \hline
4               & Словарные признаки + признаки мета-классификаторов (логистическая регрессия, ридж-регрессия, классификатор на основе градиентного бустинга и классификатор на основе нейронной сети) и линейный {\it SVM} в качестве классификатора.                                                     \\ \hline
8               & поиск эмоциональных слов по словарю (200 тыс. словоформ), правила их  комбинирования на основе синтаксического анализа; применение онтологических правил, характерных для данной предметной области                                                                                        \\ \hline
9               & {\it SVM}: униграммы, биграммы, словарь РуСентиЛекс, учет частей речи, многозначных слов (автоматический словарь коннотаций по новостям для TKK задачи)                                                                                                                                          \\ \hline
10              & {\it SVM}, в качестве признаков использовались униграммы, подвергшиеся преобразованиям ({\it не + слово = один признак}, множественные повторения символов заменяются двукратным; ссылки, ответы, даты, числа – заменяются паттернами и другие преобразования). Подключение словаря РуСентиЛекс \\ \hline
\end{tabular}
\end{table}

