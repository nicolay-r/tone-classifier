\newpage
\section{Построение классификатора для сентиментального анализа сообщений сети Twitter}
    % Цели построения классификатора, т.е. для каких сообщений он предусмотрен.

    \subsection{Обработка сообщений}
    % Векторизация, ее параметры
    Пусть имеется произвольная коллекция сообщений сети {\it Twitter}. Процесс
    обработки сообщений этой коллекции состоит из выполнения следующих этапов:
    \begin{enumerate}
        \item Лемматизация слов сообщений с целью получения списка термов;

        \item Все термы сообщения можно разделить на два класса: основные и
            дополнительные. Ко второму классу относятся следующие термы:
            \begin{itemize}
                \item Символы <<Ретвита>> --- термы со значением <<RT>>;
                \item Ссылки на ресурсы сети Интернет --- {\it URL\hspace{1pt}}-адреса;
                \item Имена пользователей сети {\it Twitter} --- термы с префиксом <<@>>;
                \item Хэштеги (от англ. {\it Hashtags}) --- термы с префиксом <<\#>>.
            \end{itemize}
            Все остальные термы относятся к классу основных, и на текущем этапе
            термы этого класса остаются в сообщении. Среди множества дополнительных
            термов, в сообщении остаются только хэштеги и {\it URL\hspace{1pt}}-адреса;

        \item Применение предварительно составленного {\it списка стоп слов} ---
            это термы, которые должны быть исключены из сообщения.
            % Описать подробнее (После)
        \item Замена некоторых биграмм и униграмм на тональные префиксы.
            % Описать подробнее (После)
    \end{enumerate}

    Рассмотрим последний пункт обработки сообщения подробнее.
    Для выполнения этого этапа, предполагается что предварительно составлен
    список пар $L_{tone} = {\langle t, s\rangle}$, где $t$ -- терм, а $s$ --
    тональная оценка (<<+>> или <<-->>). На этом этапе, для каждого терма сообщения $t_i \in L_{tone}$
    выполняется замена на соответствующую оценку $s$, которая становится префиксом
    следующего терма $t_{i+1}$. Пример преобразования:
    \begin{center}
        \it
        Сейчас \underline{хорошо} работать \underline{не} то что раньше.

        Сейчас +работать --то что раньше.
    \end{center}

    Среди описанных выше этапов обработки сообщений, обязательным и неизменным
    является только лемматизация слов сообщений. Использование и настройки
    остальных этапов могут быть изменены в зависимости от предпочтений
    пользователя.

    % Написать про tf-idf и про дополнительный словарь
    Для получения вектора на основе списка термов, подразумевается вычисление
    меры {\it tf-idf} для каждого терма сообщений обучающей коллекции.
    Дополнительно предполагатся рассмотреть {\it исскуственное} расширение исходной
    обучающей коллекции посредством информации о числе вхождений большинства
    популярных термов в корпус объемом $10^6$ сообщений. Такое расширение
    позволит оценить частоты термов на основе коллекции потенциально
    большего объема.

    \subsection{Вспомогательные признаки классификации}
    % В этот раздел вносим признаки, которые добавлялись к основной векторизации
    \subsection{Составление тестовых коллекций}
    % Здесь рассказываем про коллекции, которые использовались
    % Про балансировку коллекций в том числе.
